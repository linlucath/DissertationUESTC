% !TEX Program = xelatex
\documentclass[print, doctor, vlined]{DissertUESTC}



\begin{document}
	
	% 以下三条命令分别为高亮示例文档中的关键内容和列出所有参考文献条目而设,正式撰写时切不可使用
	\newcommand{\shad}[1]{\textcolor{DodgerBlue}{\ttfamily #1}}
	\newcommand{\shadcmd}[1]{\shad{$\backslash$#1}}
	\nocite{*}  % 为了便于在示例文档中展示参考文献而设,正式撰写时需要注释掉

	% 当论文中某节的内容接近填满页面且其下紧随几项标题时,LaTeX更倾向于在后续的标题前分页,并且纵向拉伸当前页内容的段间距,以实现纵向分散对齐,也就是很多人在问的现象。这并不是模板bug,而是LaTeX特性。
	% 出现这种情况的本质原因是用户的内容,尤其是在页面中有些图、表、标题的时候,它们的高度很可能不是正文行距的整数倍,那必然就会出现这种问题。
	% 如果你觉得Word从上到下直接堆叠内容,然后在页尾留下明显空白的处理方式更合你意,那就使用下方的\raggedbottom命令
	% \raggedbottom  % 此命令可让LaTeX像Word那样直接堆叠页面内容,而不再默认拉伸段间距,代价是页尾可能会有明显空白
	
	% \setconfidential命令用于设置论文封面中的“密级信息”。慎用!!!学生个人不应随意将论文定性为涉密,需要先经过审批。此命令必须先于\uestccover使用才有效
	% 命令参数:\setconfidential(<信息右上角与纸张左、上边界的距离,以逗号分隔>)[<字体格式>]{<密级>}{<保密期限>}。可选参数默认值:(18cm,2cm)[\zihao{4}\bfseries]
	\setconfidential{密级}{保密期限}  % 非涉密论文一定要注释掉该命令

	% 封面
%	\uestccover[<学院名称排版风格>]{<论文题目>}{<学科专业>}{<学号>}{<作者姓名>}{<指导教师>}{<教师职称>}{<学院>} % 学士、研究生通用
	\uestccover{}{}{}{}{}{}{}  % 空封面

	\uestccover{关于我的杀父仇人疑似是名震天下的大侠时该如何报仇}
				{玉女素心剑法}
				{1182000}
				{杨\hspace{1em}过}
				{小龙女}
				{掌\hspace{1em}门}%{终南山古墓派}
				{终南山古墓派终南山古墓派终南山古墓派}  % 默认将过长的学院名称压缩至下划线宽度

	\uestccover[par]{关于我的杀父仇人疑似是名震天下的大侠时该如何报仇}
				{玉女素心剑法}
				{1182000}
				{杨\hspace{1em}过}
				{小龙女}
				{掌\hspace{1em}门}%{终南山古墓派}
				{终南山古墓派终南山古墓派终南山古墓派}  % par参数将过长学院名称换行排版,此参数必影响封面布局,无可避免


	
	% 中文扉页,仅研究生用
	\DegLv{}  % 清空文档类选项设置的<申请学位级别>
	\uestczhtitlepage  % 空白中文扉页

	\ClsNum{TN828.6}  % \ClsNum{<分类号>}
	\ClsLv{公开}  % \ClsLv{<密级>}
	\UDC{621.39}  % \UDC{<UDC号>}
	\DissertationTitle{关于我的杀父仇人疑似是名震天下的大侠时该如何报仇}  % \DissertationTitle{<题名>}
	\Author{杨过}  % \Author{<作者姓名>}
	\Supervisor{小龙女}{掌门}{古墓派}{活死人墓}  % \Supervisor{<指导教师>}{<职称>}{<单位名称>}{<单位地址>}
	% 副导师信息,无则注释
	\AssociateSupervisor{洪七公}{前帮主}{丐帮}{襄阳}  % \AssociateSupervisor{<副导师名称>}{<职称}>{<单位名称>}{<单位地址>}
	\DegLv{西狂}  % \DegLv{<申请学位级别>},该信息由文档类选项自动确定,需修改默认内容时使用,否则注释即可
	\Major{玉女素心剑法}  % \Major{<学科专业>}
	\Profield{剑道}  % \Profield{专业学位领域代码},此为专业学位独有,学术学位用户注释即可
	\Date{1959年1月1日}{1961年1月1日}  % \Date{<论文提交日期>}{<论文答辩日期>}
	\Grant{中华武林}{1961年2月2日}  % \Grant{<学位授予单位>}{<学位授予日期>}
	\Reviewer{黄蓉}{一灯大师、老顽童、黄老邪、郭靖、小龙女}  % \Reviewer{<答辩委员为主席>}{<评阅人>}
	\uestczhtitlepage


	\Supervisor{小龙女}{古墓派末代掌门}{古墓派}{活死人墓}  % \Supervisor{<指导教师>}{<职称>}{<单位名称>}{<单位地址>}
	% 副导师信息,无则注释
	\AssociateSupervisor{洪七公}{天下第一帮前帮主}{丐帮}{襄阳}  % \AssociateSupervisor{<副导师名称>}{<职称}>{<单位名称>}{<单位地址>}
	\Reviewer{黄蓉}{一灯大师、老顽童、黄老邪、郭靖、小龙女、金轮法王、洪七公、欧阳锋}  % \Reviewer{<答辩委员为主席>}{<评阅人>}
	\uestczhtitlepage  % 过长评阅人文本将被自动压缩至下划线宽度,不会超出页面边界

	\uestczhtitlepage[compress]  % 在compress选项下,“中文扉页”将自动压缩过长职称的字宽,使之不超出官方模板为该内容设定的长度
	

	% 英文扉页,仅研究生用
	% \uestcentitlepage{<文题>}{<专业>}{<学号>}{<作者>}{<导师>}{<副导师>}{<学院>},若无副导师,则将“<副导师>”参数留空即可
	\uestcentitlepage{}{}{}{}{}{}{}  % 空白英文扉页
	\uestcentitlepage{How to Take Revenge When My Father's Murderer is Suspected to Be a Famous Hero}
					{Jade Lady Soul Sword Technique Jade Lady Soul Sword Technique}
					{1182000}
					{Yang Guo}
					{Grandmaster Dragondaughter Little}{Grandmaster Northern Beggar}
					{Ancient Tomb Sect Ancient Tomb Sect Ancient Tomb Sect Ancient Tomb Sect}
	
	% 独创性声明:[签名宽度]{日期}{作者签名图片}{导师签名图片},送审时使用空强制参数即可,仅研究生用
	\declaration{}{}{}  % 空白独创性声明
	
	\declaration[3cm]{2024年08月31日}{authsign}{spvrsign}

	
	
	% 开启中文摘要
	\zhabstract
	
	杨过少年时期母亲染病而亡,随后他便过着四处流浪的生活。后来遇到郭靖夫妇,便由他们照看。但之后因杨过与郭芙等人之间的矛盾,郭靖便送其去全真派习武。其后,又从全真派逃出,机缘巧合下于古墓遇见小龙女,之后便跟随小龙女练功。 他身边有许多红颜知己钟情于他,而他却一心只爱小龙女,后结为夫妇;他和郭家恩怨难分,数度因误会关系紧张,却始终挺身而出相助他们,解除嫌隙,化气为和;命运多舛,与小龙女分隔十六年里,随伴亦师亦友的神雕行侠仗义,惩恶扬善。江湖人称“神雕大侠”。 后等小龙女不至毅然跳崖殉情,在谷底与小龙女重逢后携手保卫襄阳城,杨过一展其旷世武学的威力,打败金轮法王,飞石击杀蒙古大汗,保大宋十三年和平。成为名扬天下的“神雕侠侣”。 最后一次华山论剑后,与妻子小龙女绝迹江湖。
	
	% \null\newpage

	% \null\newpage

	% 中文关键字
	\zhkeywords{练武,离经叛道,复仇,抗敌,练武,离经叛道,复仇,抗敌,练武,离经叛道,复仇,抗敌,练武,离经叛道,复仇,抗敌}
	
	
	% 开启英文摘要
	\enabstract
	
	When Yang was a teenager, his mother contracted a disease and died, and he then lived a wandering life. When he met Guo Jing and his wife, they took care of him. However, due to the conflict between Yang and Guo Fu, Guo Jing sent him to learn martial arts in the Quanzhen Sect. Later, he escaped from the Quanzhen Sect and met Xiao Longnian at the ancient tomb, where he practised kung fu with Xiao Longnian. He was surrounded by many confidantes who were in love with him, but he only loved Little Dragon Girl, and later married; he and the Guo family feud, several times due to misunderstanding tension, but always stepped forward to help them, lifting the suspicion, turning the gas into peace; ill-fated, separated from Little Dragon Girl for sixteen years, along with the Divine Eagle who is also a teacher and friend of the chivalrous, punishing the evil and promoting the good. Jianghu people ``divine eagle hero". After the little dragon lady is not to perseverance to jump off the cliff to martyrdom, in the bottom of the valley and the little dragon lady reunited with the defence of Xiangyang City, Yang past a show of its unparalleled martial arts power to defeat the Golden Wheel of the Fa Wang, flying stone to kill the Mongolian Khan, to protect the thirteen years of peace in the Great Song Dynasty. He became the world-famous ``Divine Eagle Couple". After the last Mount Hua sword debate, he and his wife Xiaolongnu went into exile.
	
	
	% 英文关键字
	\enkeywords{Martial arts, apostasy, revenge, fighting against the enemy, martial arts, apostasy, revenge, fighting against the enemy}
	
	\tableofcontents  % 主目录,必要
	
	\listoffigures  % 图多则放,反之不放
	
	\listoftables  % 表多则放,反之不放
	
	\listofsymbs  % 生成主要符号表标题,需要额外维护符号表内容
	
	生成主要符号表的章标题需使用本模板提供的\shadcmd{listofsymbs}命令。排版主要符号表的内容则需要使用本模板提供的\shad{symbtable}环境。该环境基于\shad{longtable}环境进行封装,依次接受两个可选参数:
	
	\shadcmd{begin\{symbtable\}[<表格整体位置>](<主要符号表的列控制参数>)}
	
	\noindent 其中,第一项可选参数用于设置\shad{longtable}环境的可选参数,且默认值与\shad{longtable}环境保持一致;第二项可选参数用于设置\shad{longtable}环境的必选参数,其默认值设置为\shad{p\{3.5em\} p\{$\backslash$linewidth-9em\} p\{3em\}<\{$\backslash$centering\}}。
	
	若非必要,用户不应指定\shad{symbtable}环境的可选参数。但若出于对排版美观性的考虑,可适当调整主要符号表各列的宽度。注意,按照学位论文撰写规范中的示例,\textbf{主要符号表有且仅有三列}。因此,切勿对第二项可选参数设置其他列数。
	
	\textbf{重要提醒}:\shadcmd{listofsymbs}命令和\shad{symbtable}环境必须同时出现或消失。消失好理解,不需要主要符号表的时候,通通注释掉即可;需要的时候,必须使用\shad{symbtable}环境生成表格内容,因为在印刷模式下,这部分结束时是否需要添加左手空白页的判断逻辑放在了\shad{symbtable}环境结束时(自动)执行。
	
	\begin{symbtable}
		a & 加速度(acceleration) & 1 \\
		A & 振幅(amplitude)、面积(Area)、磁场矢量势(magnetic vector potential) & 2 \\
		B & 磁场、磁感应强度、核结合能 & 3 \\
		c & 真空中光速 & 4 \\
		C & 比热容(heat capacity)、电容 & 5 \\
		d & 长度(distance)、直径(diameter)、微分(differential,如dx)& 6 \\
		D & 电位移矢量(electric displacement) & 7 \\
		e & 元电荷、自然底数(欧拉常数)& 8 \\
		E & 能量、电场场强,电动势 & 9 \\
		f & 频率、焦距(光学) & 10 \\
		F & 力、通量(Flux) & 11 \\
		g &(地表)重力加速度 & 12 \\
		G & 万有引力常数 & 13 \\
		h & 普朗克常数、高度 & 14 \\
		H & 哈勃常数、焓、磁化强度矢量、哈密顿算符(Hamiltonian) & 15 \\
		i & 虚数单位 & 16 \\
		I & 电流、惯量(inertia)、冲量(impulse) & 17 \\
		j & 辐射强度、加加速度(jerk) & 18 \\
		J & 角动量、概率流(量子力学)、电流密度、巨配分函数里的巨势 Z(J) & 19 \\
		k & 玻尔兹曼(Boltzmann)常数、库伦常数、用来指代某常量或固定比值 & 20 \\
		K & 四维波矢量(相对论)、动能 & 21 \\
		l & 长度(length) & 22 \\
		L & 角动量、电感系数 & 23 \\
		m & 质量 & 24 \\
		M & 磁化率 & 25 \\
		n & 序数、主量子数、摩尔数(化学)、折射率(光学) & 26 \\
		N & 序数、中子数、放大倍率(光学)& 27 \\
		o & 小o符号 & 28 \\
		O & 大O符号 & 29 \\
		p & 动量、压强(pressure)、电偶极矩(electric dipole moment) & 30 \\
		P & 概率(量子力学,统计学)、功率(power)、极化度 & 31 \\
		q & 电荷 & 32 \\
		Q & 电荷热量、流量 & 33 \\
		r & 半径、位置向量、球坐标系半径轴 & 34 \\
		R & 电阻、普适气体常数(热力学)、里德伯(Rydberg)常数(光谱学)、反射率(Reflectivity,光学) & 35 \\
		s & 自旋 & 36 \\
		S & 熵、面积 & 37 \\
		t & 时间 & 38 \\
		T & 温度、周期、透射率(Transmittance,光学) & 39 \\
		u & 原子质量单位、物距(光学) & 40 \\
		U & 电压、电势 & 50 \\
		v & 速度、像距(光学) & 51 \\
		V & 体积、势能 & 52 \\
		w & 速度 & 53 \\
		W & 功 & 54 \\
		x & 直角坐标系横轴 & 55 \\
		X & 磁化率、电抗 & 56 \\
		y & 直角坐标系纵轴 & 57 \\
		Y & 光亮度、球谐函数 & 58 \\
		z & 直角/圆柱坐标系竖轴、复数变量 & 59 \\
		Z & 阻抗、原子序数(质子数)、配分函数(partition function) & 60 \\
		$\alpha$ & 角度、精细结构常数、角加速度、四维加速度(相对论)、攻角 & 61 \\
		Α & Alpha (与英文拼写难区分的希腊字母一般不使用) & 62 \\
		$\beta$ & 角度;磁通系数;速度与光速的比值 & 63 \\
		Β & Beta Β函数 & 64 \\
		γ & 电导系数、洛伦兹因子、热容比 & 65 \\
		Γ & Gamma Γ函数、克里斯托弗尔符号 & 66 \\
		δ & 狄拉克δ函数、克罗内克函数(Kronecker delta)、屈光度、微分 & 67 \\
		Δ & Delta 拉普拉斯算子、有限差 & 68 \\
		ε & 电容率(permittivity)、介电常数、列维-奇维塔符号(Levi-Civita symbol)、发射率 & 69 \\
		Ε & Epsilon & 70 \\
		Ϝ & digamma & 71 \\
		ζ & 阻尼比、黎曼ζ函数 & 72 \\
		Ζ & Zeta & 73 \\
		η & 黏度(viscosity)、磁滞系数、效率 & 74 \\
		Η & Eta & 75 \\
		θ & 角变量、温度、球坐标/圆柱坐标横角 & 76 \\
		Θ & Theta 单位阶跃函数(Heaviside step function) & 77 \\
		Ι & Iota & 78 \\
		$\kappa$ & 介质常数比ε/ε0、导热率、热扩散率 & 79 \\
		$\kappa$ & Kappa & 80 \\
		λ & 波长、mean free path、半衰期 & 81 \\
		$\Lambda$ & Lambda 洛伦兹变换、冯·卡门常数 & 82 \\
		μ & 磁导率、质子质量单位、摩擦系数、离子迁移率 & 83 \\
		Μ & Mu & 84 \\
		ν & 频率、运动粘度、自由度 & 85 \\
		Ν & Nu & 86 \\
		ξ & 黎曼ξ函数、随机变量 & 87 \\
		Ξ & Xi & 88 \\
		Ο & Omicron & 89 \\
		π & 圆周率、共轭动量 & 90 \\
		∏ & Pi 乘积 & 91 \\
		ρ & 密度、电阻率 & 92 \\
		Ρ & Rho & 93 \\
		σ & 导电率、斯提芬-波尔兹曼常数(Stefan–Boltzmann constant)、核反应截面、表面密度、标准差 & 94 \\
		∑ & Sigma 总和 & 95 \\
		τ & 扭矩(Torque)、剪应力(Shear stress)、时间常数、2π & 96 \\
		Τ & Tau & 97 \\
		Υ & Upsilon & 98 \\
		φ & 球坐标纵角、波相位(wave phase)、直径 & 99 \\
		Φ & Phi 磁通量、辐射通量、逸出功 & 100 \\
		χ & 电极化率(Electric susceptibility) & 101 \\
		Χ & Chi 电抗 & 102 \\
		ψ & 角速;介质电通量(静电力线) & 103 \\
		Ψ & Psi 波函数 & 104 \\
		ω & 角速度 & 105 \\
		Ω & omega 蔡廷常数(Chaitin's constant) 立体角(Solid angle) & 106 \\
	\end{symbtable}
	
	
	
	% 打印缩略词表,\printnomenclature[<英文缩写宽度>](<中文全称宽度>)
	\printnomenclature
	\nomchn{LP}{Linear Programming}{线性规划}  % 创建缩略词条目,% \nomchn{<缩略词>}{<英文全称>}{<中文全称>}
	\nomchn{PLE}{Path Loss Exponent}{路径损失指数}
	\nomchn{QoS}{Quality of Service}{服务质量}
	\nomchn{SLA}{Service Level Agreement}{服务水平协议}
	\nomchn{NLP}{non-linear programming}{非线性规划}
	\nomchn{4G}{Fourth Generation Mobile Communication Technology}{第四代移动通信技术}
	\nomchn{5G}{Fifth Generation Mobile Communication Technology}{第五代移动通信技术}
	\nomchn{B5G}{Beyond 5G}{超五代移动通信技术}
	\nomchn{NSA}{Non-Standalone}{非独立组网}
	\nomchn{mMIMO}{massive Multiple Input Multiple Output}{大规模多输入多输出}
	\nomchn{FDMA}{Frequency Division Multiple Access}{频分多址}
	\nomchn{TDMA}{Time Division Multiple Access}{时分多址}
	\nomchn{IP}{Internet Protocol}{网际互连协议}
	\nomchn{CDMA}{Code Division Multiple Access}{码分多址}
	\nomchn{LTE}{Long Term Evolution}{长期演进}
	\nomchn{OFDM}{Orthogonal Frequency Division Multiplexing}{正交频分复用}
	\nomchn{SDMA}{Space Division Multiple Access}{空分多址}
	\nomchn{MIMO}{Multiple Input Multiple Output}{多输入多输出}
	\nomchn{MBS}{Macro Base Station}{宏基站}
	\nomchn{SBS}{Small Base Station}{微基站}
	\nomchn{PBS}{Pico Base Station}{微微基站}
	\nomchn{FBS}{Femto Base Station}{毫微微基站}
	\nomchn{NR}{New Radio}{新空口}
	\nomchn{EPC}{Evolved Packet Core}{演进分组核心}
	\nomchn{5GC}{The Fifth Generation Core Network}{5G核心网}
	\nomchn{RAN}{Radio Access Network}{无线接入网络}
	\nomchn{Multi-RAT}{Multi Radio Access Technology}{多制式}
	\nomchn{HetNet}{Heterogeneous Network}{异构网络}
	\nomchn{CSI}{Channel State Information}{信道状态信息}
	\nomchn{3GPP}{3rd Generation Partnership Project}{第三代合作伙伴计划}
	\nomchn{PDCP}{Packet Data Convergence Protocol}{分组数据汇聚协议}
	\nomchn{MME}{Mobile Management Entity}{移动管理实体}
	\nomchn{S-GW}{Serving GateWay}{服务网关}
	\nomchn{RLC}{Radio Link Control}{无线链路控制协议}
	\nomchn{MAC}{Media Access Control}{媒体访问控制协议}
	\nomchn{SDN}{Software-Defined Networking}{软件定义网络}
	\nomchn{NFV}{Network Functions Virtualization}{网络功能虚拟化}
	\nomchn{NSI}{Network Slice Instance}{网络切片实例}
	\nomchn{MANO}{Management and Network Orchestration}{管理和网络编排}
	\nomchn{SDM-X}{Software-Defined Mobile Network Coordinator}{软件定义移动网络协调器}
	\nomchn{SDM-C}{Software-Defined Mobile Network Controller}{软件定义移动网络控制器}
	\nomchn{VNF}{Virtual Network Function}{虚拟网络功能}
	\nomchn{PNF}{Physical Network Function}{物理网络功能}
	\nomchn{ETSI}{European Telecommunications Standards Institute}{欧洲电信标准化协会}
	\nomchn{NFVO}{Network Function Virtualization Orchestrator}{虚拟网络功能编排器}
	\nomchn{VNFM}{Virtual Network Function Manager}{虚拟网络功能管理器}
	\nomchn{VIM}{Virtual Infrastructure Manager}{虚拟基础设施管理器}
	\nomchn{NSMF}{Network Slice Management Function}{网络切片管理功能}
	\nomchn{NSSMF}{Network Sub-Slice Management Function}{网络子切片管理功能}
	\nomchn{SDM-O}{Software-Defined Mobile Network Orchestrator}{软件定义移动网络编排器}
	\nomchn{SLA}{Service Level Agreement}{服务水平协议}
	\nomchn{5GPPP}{5G Infrastructure Public Private Partnership}{5G基础设施公私合作伙伴关系}
	\nomchn{InP}{Infrastructure Provider}{基础设施提供商}
	\nomchn{QoS}{Quality of Service}{服务质量}
	\nomchn{NFC}{Network Function Component}{网络功能组件}
	\nomchn{QoE}{Quality of Experience}{服务体验}
	\nomchn{AI}{Artificial Intelligence}{人工智能}
	\nomchn{RSRP}{Reference Singal Receiving Power}{参考信号接收功率}
	\nomchn{RSSI}{Received Signal Strength Indicator}{接收信号强度指标}
	
	生成缩略词表相对复杂一些:
	\begin{enumerate}
		\item 先使用\shadcmd{printnomenclature[<英文缩写宽度>](<中文全称宽度>)},第一项可选参数控制\textbf{英文缩写}的列宽,默认为\shad{5em};第二项可选参数控制\textbf{中文全称}的列宽,默认为\shad{7.5em}。
		
		\item 然后在正文中出现缩略词的位置使用命令\\\shadcmd{nomchn[<排序前缀>]\{<缩略词>\}\{<英文全称>\}\{<中文全称>\}}添加该缩略词条目。其中\shad{排序前缀}是可选参数,仅在对特定条目有特殊排序需求时才使用。具体细节参考\href{https://mirrors.hust.edu.cn/CTAN/macros/latex/contrib/nomencl/nomencl.pdf}{\shad{\color{DarkRed}nomencl}}宏包对\shadcmd{nomenclature}命令的参数说明。
	\end{enumerate}
	
	另外,本地用户需要先编译生成缩略词表的辅助文件,再编译完整文档才能获得正确的结果,教程参见\href{https://zhuanlan.zhihu.com/p/46442713}{\shad{编译缩略词表}}。Overleaf用户则可以一键搞定,无需额外操作。
	
	细心的朋友可能会发现,在印刷模式下,当前段落所在的页并未标注任何页眉与页脚,猜猜为什么。提示一下,当你正式使用本模板撰写论文时,肯定不需要此页的内容。

	
	\chapter{模板使用说明}
	
	\section{导言区及模板选项}

	模板的导言区只有两行:
	\begin{itemize}
		\item \shad{\% !TEX Program = xelatex}在Texstudio中表示指定使用XeLaTeX编译该文档,对其他编辑器,可能需要用户手动设置编译引擎。

		\item \shadcmd{documentclass[<选项列表>]\{DissertUESTC\}}表示加载名为\shad{DissertUESTC}的文档类,该文档类基于LaTeX的\shad{book}类编写。此文档类新增\textbf{\color{DarkRed}七种}选项:

		\begin{itemize}
			\item \shad{print}/\shad{nonprint}:该选项控制是否以印刷模式生成文档,印刷模式会自动在论文的前置部分添加必要的空白页。默认为\shad{print}。

			\item \shad{doctor}/\shad{prodoctor}/\shad{intdoctor}/\shad{master}/\shad{promaster}/\shad{intmaster}/\shad{bachelor}:该选项设置学位论文类型,分别对应学术学位博士、专业学位博士、International Doctor、学术学位硕士、专业学位硕士、International Master以及学士学位。默认为“\shad{doctor}”。
   
			\item \shad{subfigsimple}/\shad{subfigparens}:该选项用于调整正文中对子图标签进行引用生成的编号样式,\shad{subfigsimple}对应样式为\shad{1-1a},\shad{subfigparens}对应样式为\shad{1-1(a)},默认为\shad{subfigparens}。
   
			\item \shad{draftfig}:LaTeX标准文档类提供的\shad{draft}选项在排版草稿时不会生成交叉引用链接、超链接、书签,图片也会被替换为尺寸与之相同的方框+文本,并且会在超出表格、页面边界的位置标注粗框线。\shad{draftfig}选项则仅将图片替换为方框+文本,而不修改标准\shad{draft}选项涉及的其他内容。其主要用处是在自行查重时便捷地隐去论文中的图片,而不影响排版。
			
			\item \shad{review}:此选项将以评审模式排版论文的封面及中英文扉页,届时所有能确定个人身份的信息都将被隐去,包括导师信息以及独创性声明中的签名和日期(虽然在送审的时候也不会有这两个信息)。当然,你也可以采用本文档后续介绍的设置空参数的方法来隐去对应信息,但此选项能让你在不调整命令参数内容的情况下实现同样的效果。我个人觉得会更方便一点。
			
			\item \shad{noreminder}:默认情况下,当\textbf{中文摘要}和\textbf{致谢}的篇幅超出规范的最大页数限制时,模板将在对应内容的结尾显式打印提醒信息。若用户在知悉这些内容的长度超出规范限制后仍希望保持原样,则可使用\shad{noreminder}选项禁用提醒信息。(2025.02.22)
			
			\item \shad{cmmmath}/\shad{timesmathnogreek}/\shad{timesmath}:千呼万唤始出来,该选项用于选择渲染公式使用的字体。其中,\shad{cmmmath}即对应LaTeX原本使用的Computer Modern Math。这是添加此类选项前,本模板默认使用的公式字体,也是我个人比较喜欢的,现在仍然是此类选项的默认值;\shad{timesmathnogreek}指定使用Times New Roman来渲染公式中的英文字母和数字,但不会更改希腊字母、手写体和双线体的字体;\shad{timesmath}则同时将希腊字母也设置成Times New Roman,手写体和双线体仍保持不变,不过我觉得希腊字符用这个字体并不好看。对公式字体是不是均为Times New Roman比较介怀的用户可以使用\shad{timesmath}选项。后两种选项均基于\href{https://mirrors.nju.edu.cn/CTAN/macros/xetex/latex/mathspec/mathspec.pdf}{\shad{\color{DarkRed}mathspec}}宏包实现,在我有限的测试实践中,只有它能做到真正意义上的Times New Roman。(2025.01.31)
			
			\textbf{PS1:}需要注意,后两种选项使用的Times New Roman字体本不支持在公式中排版粗斜体,即原本的\shad{$\backslash$boldsymbol\{\}}命令会失效。为了解决这个问题,我(仅在后两种选项下)对这条命令进行了粗糙的重定义,使之能像原版那样生成粗斜体符号。重定义后的\shad{$\backslash$boldsymbol\{\}}命令需要遵循一条额外的使用规则:\textbf{其输入参数必须是最原始的数学符号}。比如你想排版\shad{$\backslash$boldsymbol\{$\backslash$hat\{$\backslash$alpha\}\}}(这在\shad{cmmmath}下是没有问题的),那此时正确的源码应该是\shad{$\backslash$hat\{$\backslash$boldsymbol\{$\backslash$alpha\}\}},\textbf{即将\shad{$\backslash$boldsymbol\{\}}置于嵌套的最内层}。如若不然,模板轻则无法渲染出预期的数学符号(在\shad{timesmath}选项下),重则直接报错(在\shad{timesmathnogreek}选项下)。大概是涉及了一些底层的问题,我也不懂。

			\textbf{PS2:}因为\href{https://mirrors.nju.edu.cn/CTAN/macros/xetex/latex/mathspec/mathspec.pdf}{\shad{\color{DarkRed}mathspec}}宏包本身的特性,使用Times New Roman作为公式字体需要用户付出更多精力。举个例子,你想排版\shad{\$f\^{}t\$},那么你会发现\shad{f}和\shad{t}之间的间隔很小,两者重叠了,这时候需要你手动用\shad{"}插入空格,变成\shad{\$f\^{}\{"t\}\$}。使用\shad{timesmathnogreek}和\shad{timesmath}选项的用户均很容易遇到这类问题,届时就需要仔细查阅\href{https://mirrors.nju.edu.cn/CTAN/macros/xetex/latex/mathspec/mathspec.pdf}{\shad{\color{DarkRed}mathspec}}的宏包文档。

			\textbf{PS3:}论文撰写规范其实并未对公式使用的字体作强制要求,而且已经有同学使用本模板之前的版本通过了学校的格式审查。他提供的信息是说:审查系统会识别到公式字体不是Times New Roman,但级别是提醒而非错误,不会造成格式审查不通过。然则,实在有太多人问怎么公式不是Times New Roman了,既然有那么多人喜欢Times New Roman,那秉承本模板一贯的行事风格,选择权交给你自己。

			\item 另外,\href{https://mirrors.sustech.edu.cn/CTAN/macros/latex/contrib/algorithm2e/doc/algorithm2e.pdf}{\shad{\color{DarkRed}algorithm2e}}宏包的\shad{vlined}和\shad{boxruled}选项也能通过文档类设置。
		\end{itemize}
	\end{itemize}
	
	% \newpage
	\section{各级标题}
	
	本模板基于\shad{book}类,章标题需要使用\shad{$\backslash$chapter\{<章标题>\}} 生成,其他各级标题依次为\shad{$\backslash$section\{<节标题>\}}、\shad{$\backslash$subsection\{<子节标题>\}}、\\ \shad{$\backslash$subsubsection\{<孙节标题>\}}。
	
	\section{图片}
	
	本模板使用graphicx和subfig宏包来处理插入的图片及子图,需要将待排版图片文件放入项目目录\shad{./fig/}中。以下给出一些排版图片的例子。
	
	\begin{figure}[!htb]
		\centering
		\includegraphics[width=0.6\linewidth]{黄蓉郭靖1}
		\caption{锁定仇人}
	\end{figure}
	
	% \clearpage
	\begin{figure}[!htb]
		\centering
		\subfloat[]{
			\includegraphics[width=0.4\linewidth]{杨过小龙女3}
			\label{fig: 见到姑姑嘻嘻}
		}
		\hfill
		\subfloat[]{
			\includegraphics[width=0.4\linewidth]{杨过小龙女6}
			\label{fig: 姑姑见我不嘻嘻}
		}
		\caption[报仇哪有姑姑重要]{报仇哪有姑姑重要。\subref{fig: 见到姑姑嘻嘻}见到姑姑我嘻嘻;\subref{fig: 姑姑见我不嘻嘻}姑姑见我不嘻嘻} \label{fig: 报仇哪有姑姑重要}
	\end{figure}
	

	需要注意,图\ref{fig: 报仇哪有姑姑重要}中引用子图\ref{fig: 见到姑姑嘻嘻}和本段中引用子图使用的命令分别为\shad{$\backslash$subref\{fig: 见到姑姑嘻嘻\}}和\shad{$\backslash$ref\{fig: 报仇哪有姑姑重要\}},它们分别生成仅含带括号子图编号和完整子图编号的结果。
	
	另外,图\ref{fig: 报仇哪有姑姑重要}的图题包含了子图题文本,但生成的图目录中却只有主图题文本,其实现方式为在主图题命令中使用可选参数单独指定图目录中的显示文本:\shad{$\backslash$caption[报仇哪有姑姑重要]\{<实际图题>\}}
	
	\begin{figure}[!htb]
		\centering
		\subfloat[]{
			\includegraphics[width=0.4\linewidth]{陆无双2}
			\label{fig: 陆无双2}
		}
		\hfill
		\subfloat[]{
			\includegraphics[width=0.4\linewidth]{程英3}
			\label{fig: 程英3}
		}
		\caption{找其他红颜知己嘻嘻。\subref{fig: 陆无双2}眼睛像姑姑;\subref{fig: 程英3}举止像姑姑} \label{fig: 红颜知己}
	\end{figure}
	
	\begin{figure}[!htb]
		\centering
		\subfloat[]{
			\includegraphics[width=0.4\linewidth]{绿萼2}
			\label{fig: 绿萼2}
		}
		\hfill
		\subfloat[]{
			\includegraphics[width=0.4\linewidth]{杨过绿萼}
			\label{fig: 杨过绿萼}
		}
		\\
		\subfloat[]{
			\includegraphics[width=0.98\linewidth]{陆无双程英}
			\label{fig: GG}
		}
		\caption{撩妹是我杨过的被动技能。\subref{fig: 绿萼2}好腼腆的姑娘;\subref{fig: 杨过绿萼}你终于肯笑了;\subref{fig: GG}哦吼} \label{fig: 被动技能}
	\end{figure}

	研究生和本科生学位论文规范对多行图题左右侧缩进距离的要求不同,前者为单侧\shad{4em},后者为单侧\shad{2em}。此参数由论文类型选项控制,无需用户过问。各位可以试试看,图\ref{fig: 被动技能}的主图题在\shad{bachelor}选项下能单行排版,而在其他类型选项下会换行。
	
	\begin{figure}[!htb]
		\centering
		\includegraphics[width=0.98\linewidth]{杨过郭靖}
		\caption{还是推主线吧,动手动手}
	\end{figure}
	
	\clearpage
	\section{表格}

	\textbf{写在最开始}:由于一些实现上的问题,对于确定非置底排版的任何表格,用户在通过\shad{table}环境的选项指定可选择的排版模式时,\textbf{不可提供\shad{b}模式},否则表格的上间距将过窄;反之,对于一页内确定置底排版的第一个表格,用户需要\textbf{显式在选项中指定\shad{b}或\shad{!b}},否则此表格上间距将过宽。

	若出现意料之外的情况,用户可以通过在\shad{table}环境开始后和结束前的位置插入\shad{$\backslash$vspace*\{<距离长度>\}}来调整其上下间距,使之看起来协调。
	
	\subsection{普通表格}
	
	普通表格的排版本身无需多言,使用\shad{table}+\shad{tabular}环境即可,但是要注意三线表中的三条线分别需要使用\shad{$\backslash$toprule}、\shad{$\backslash$midrule}、\shad{$\backslash$bottomrule}生成,这样才符合研究生规范中对线高的要求(\shad{1.5}磅、\shad{0.75}磅、\shad{1.5}磅)。注意不要用\shad{$\backslash$hline}。而对于本科生,学士学位论文规范要求表格中的线高统一为\shad{0.5}磅,在\shad{bachelor}选项下,\shad{$\backslash$midrule}的线高设置为\shad{0.5}磅。因此,本科生在表格中只需要也只可以使用\shad{$\backslash$midrule}和\shadcmd{cmidrule},否则线高将不符合要求。
	
	在需要为表格中的某些单元格添加水平框线时,应使用\newline\shadcmd{cmidrule[<线高>](<修剪>)\{<起始列-终止列>\}}而非\shadcmd{cline\{<起始列-终止列>\}}。后者似乎无法调整线高,也无法对框线的端点进行修剪。前者的第一项可选参数允许用户设置框线高度,其默认值在\shad{bachelor}选项下设置为了\shad{0.5}磅,而在其他论文类型下设置为了\shad{0.75}磅。如非必要,用户无需设置该可选参数;第二项可选参数允许用户对框线的端点进行修剪,当需要避免同行独立的相邻框线在视觉上连通到一起时,该选项将很有用,比如表\ref{tab: cmidrule示例}中的示例。有关第二项可选参数可取的值,建议用户查阅\href{https://mirrors.sustech.edu.cn/CTAN/macros/latex/contrib/booktabs/booktabs.pdf}{\shad{\color{DarkRed}booktabs}}宏包的官方文档。
	
	\begin{table}[htp]
		\caption{$\backslash$cmidrule示例}\label{tab: cmidrule示例}
		\begin{threeparttable}
			\begin{tabular}{ccccc}
				\toprule
				\multirow{2}{*}{Column0} &  \multicolumn{2}{c}{Column1\tnote{1}} & \multicolumn{2}{c}{Column2\tnote{2}} \\
				\cmidrule(lr){2-3}\cmidrule[2.5bp](l){4-5}
				~     & subcolumn1 & subcolumn2 & subcolumn1 & subcolumn2 \\
				\midrule
				Row1  & element11 & element12 &element13 & element14 \\
				Row2  & element21 & element22 &element23 & element24 \\
				\cmidrule{2-3}\cmidrule[2.5bp]{4-5}
				Row3  & element31 & element32 &element33 & element34 \\
				\bottomrule
			\end{tabular}
			\begin{tablenotes}
				\item[1] 在\shad{bachelor}选项下,\shadcmd{cmidrule}默认线高设置为0.5bp,而在其他论文类型下,默认值为0.75bp,两者规范的要求不同。可用第二项可选参数同时修剪掉框线的左右端点
				\item[2] 通过指定\shadcmd{cmidrule}的第一项可选参数调整线高,并用第二项可选参数仅修剪掉框线的左端点
			\end{tablenotes}
		\end{threeparttable}
	\end{table}

	
	\newpage
	\subsection{带附注表格}
	
	更需要说明的是生成带附注的表格。本模板采用\shad{threeparttable}宏包实现将表格中的附注内容顶格排版在表格底部:
	\begin{enumerate}
		\item 使用\shadcmd{tnote\{<label>\}}在表格中插入上标编号;
		\setcounter{enumi}{98}
		\item 使用\shad{tablenotes}环境在表格底部排版附注。该环境提供选项\shad{online}用于将附注文本前的标号从默认的上标样式(见表\ref{tab: 江湖势力背调})更改为非上标样式(见表\ref{tab: 已习得武功})。
	\end{enumerate}
	
	\begin{table}[!ht]
		\caption{江湖势力背调} \label{tab: 江湖势力背调}
		\begin{threeparttable}
			\begin{tabular}{p{2cm} p{3cm} p{7cm}}
				\toprule
				\textbf{姓名} & \textbf{所属势力} & \textbf{武功绝学} \\
				\midrule
				郭靖 & 重阳宫 & 降龙十八掌 \\
				黄蓉 & 丐帮 & 打狗棒法 \\
				洪七公 & 丐帮 & 降龙十八掌、打狗棒法 \\
				黄老邪 & 桃花岛 & 弹指神通、落英神剑掌、玉箫剑法 \\
				老顽童 & 重阳宫 & 左右互博术\tnote{1} \\
				一灯 & 云南大理 & 一阳指\tnote{2}、千里传音 \\
				\bottomrule
			\end{tabular}
			\begin{tablenotes}
				\item[1] 左右互搏术是金庸小说《射雕英雄传》中「老顽童」周伯通在桃花岛的地洞中创出的武功,本质是一心二用,能够两手同时做不同的事情,在金庸武侠体系中是一门非常精妙的武学,其对于人物本身的战斗力加成堪称台阶性。
				\item[2] 云南大理段氏嫡传的武功,在点穴功夫中位居天下第一,运功后以右手食指点穴,出指可缓可快,缓时潇洒飘逸,快则疾如闪电,但着指之处,分毫不差。当与敌挣搏凶险之际,用此指法既可贴近径点敌人穴道,也可从远处欺近身去,一中即离,一攻而退,实为克敌保身的无上妙术。
			\end{tablenotes}
		\end{threeparttable}
	\end{table}
	
	\begin{table}[!ht]
		\caption{已习得武功} \label{tab: 已习得武功}
		\begin{threeparttable}
			\begin{tabular}{p{3cm} p{3cm} p{5cm}}
				\toprule
				\textbf{武功绝学} & \textbf{传授者} & \textbf{传授地点} \\
				\midrule
				蛤蟆功 & 欧阳锋 & 重阳山脉 \\
				九阴真经 & 小龙女 & 活死人墓 \\
				打狗棒法 & 洪七公、黄蓉 & 华山之巅、英雄大会 \\
				玉箫剑法 & 黄老邪 & 深山老林 \\
				黯然销魂掌\tnote{1} & 自创 & 海边 \\
				\bottomrule
			\end{tabular}
			\begin{tablenotes}[online]
				\item[1] 黯然销魂掌,是在杨过与小龙女离别后,认为今生再也见不到小龙女,悲从中来,由此创作了黯然销魂掌。黯然销魂掌和心情有关,此后杨过与小龙女重逢后,其心理愉悦,故使不出黯然销魂掌。
			\end{tablenotes}
		\end{threeparttable}
	\end{table}

	上述方式排版的带附注表格无法通过点击表格中的编号跳转到对应附注。为此,本模板提供命令\shadcmd{puttablenotelabel\{<标签>\}}和\shadcmd{tablenoteref\{<标签>\}}来实现该操作。\textcolor{red}{(2025.01.05新增)}

	用户只需要将\shad{tablenotes}环境中手动设置的编号替换为\shadcmd{puttablenotelabel\{<标签>\}},然后在表格内容的对应位置使用\shadcmd{tablenoteref\{<标签>\}}即可。编号将自动生成,并按照使用\shadcmd{puttablenotelabel}的顺序递加。

	这种方式由于需要建立交叉引用,通常需要用户编译两次。切记,\textbf{标签必须全文唯一}。表\ref{tab: 江湖势力背调(基于puttablenotelabel和tablenoteref)}提供了使用示例。

	\begin{table}[!ht]
		\caption{江湖势力背调(基于\shadcmd{puttablenotelabel}和\shadcmd{tablenoteref})} \label{tab: 江湖势力背调(基于puttablenotelabel和tablenoteref)}
		\begin{threeparttable}
			\begin{tabular}{p{2cm} p{3cm} p{7cm}}
				\toprule
				\textbf{姓名} & \textbf{所属势力} & \textbf{武功绝学} \\
				\midrule
				郭靖 & 重阳宫 & 降龙十八掌 \\
				黄蓉 & 丐帮 & 打狗棒法 \\
				洪七公 & 丐帮 & 降龙十八掌、打狗棒法 \\
				黄老邪 & 桃花岛 & 弹指神通、落英神剑掌、玉箫剑法 \\
				老顽童 & 重阳宫 & 左右互博术\tablenoteref{tn: 左右互搏术} \\
				一灯 & 云南大理 & 一阳指\tablenoteref{tn: 一阳指}、千里传音 \\
				\bottomrule
			\end{tabular}
			\begin{tablenotes}
				\item[\puttablenotelabel{tn: 左右互搏术}] 左右互搏术是金庸小说《射雕英雄传》中「老顽童」周伯通在桃花岛的地洞中创出的武功,本质是一心二用,能够两手同时做不同的事情,在金庸武侠体系中是一门非常精妙的武学,其对于人物本身的战斗力加成堪称台阶性。
				\item[\puttablenotelabel{tn: 一阳指}] 云南大理段氏嫡传的武功,在点穴功夫中位居天下第一,运功后以右手食指点穴,出指可缓可快,缓时潇洒飘逸,快则疾如闪电,但着指之处,分毫不差。当与敌挣搏凶险之际,用此指法既可贴近径点敌人穴道,也可从远处欺近身去,一中即离,一攻而退,实为克敌保身的无上妙术。
			\end{tablenotes}
		\end{threeparttable}
	\end{table}
	
	
	\clearpage
	\subsection{跨页表格}
	
	原则上,长度不足一页的表格不应跨页。而对于本身超过一页的表格,本模板使用\href{https://mirrors.tuna.tsinghua.edu.cn/CTAN/macros/latex/required/tools/longtable.pdf}{\shad{\color{DarkRed}longtable}}宏包提供的\shad{longtable}环境实现。用户需要了解\shad{longtable}环境的基本使用方法,它与\shad{tabular}环境的最大区别在于需要用户自行定义分页后的表题、表头以及表尾。本模板提供了命令\shadcmd{CPcaption\{<当前表格总列数>\}\{<跨页表题>\}}来正确排版跨页之后的表题\textcolor{red}{(2025.01.05日更新该命令的使用方式)}。\textbf{务必使用此命令},否则跨页后的表题将会与表格内容采用相同的行距和段前段后,而非与规范中要求的表题格式保持一致;并且在跨页表题长度超过表宽时,无法产生预期的排版结果。
	
	此外,不应将\shad{longtable}环境嵌套在\shad{table}等浮动环境中,否则长表格将无法正常跨页。具体细节参见本小节示例表\ref{tab: 中国计算机学会部分推荐期刊及会议}和表\ref{tab: 中国计算机学会部分推荐期刊及会议简表}。
	
	% \newpage

	\begin{longtable}{p{2em} p{4.5em} p{11em} p{6em} p{11em}}
		\caption{中国计算机学会部分推荐期刊及会议} \label{tab: 中国计算机学会部分推荐期刊及会议} \\
		
		\toprule
		\textbf{序号} & \textbf{刊物简称} & \textbf{刊物全称} & \textbf{出版社} & \textbf{网址} \\
		\midrule
		\endfirsthead
		
		% 在这里设计首页以外的表题和表头
		\CPcaption{5}{中国计算机学会部分推荐期刊及会议}\\
		\toprule
		\textbf{序号} & \textbf{刊物简称} & \textbf{刊物全称} & \textbf{出版社} & \textbf{网址} \\
		\midrule
		\endhead
		
		% 在这里设计首页以外的表尾
		\bottomrule
		\multicolumn{5}{l}{续下页} \\  % 如不希望跨页表尾显示任何内容则注释掉即可
		\endfoot
		
		\bottomrule
		\endlastfoot
		
		1 & JSAC & IEEE Journal on Selected Areas in Communications & IEEE & http://dblp.uni-trier.de/db/journals/jsac/ \\
		2 & TMC & IEEE Transactions on Mobile Computing & IEEE & http://dblp.uni-trier.de/db/journals/tmc/ \\
		3 & TON & IEEE/ACM Transactions on Networking & IEEE/ACM & http://dblp.uni-trier.de/db/journals/ton/ \\
		1 & TOIT & ACM Transactions on Internet Technology & ACM & http://dblp.uni-trier.de/db/journals/toit/ \\
		2 & TOMM & ACM Transactions on Multimedia Computing, Communications and Applications & ACM & http://dblp.uni-trier.de/db/journals/tomccap/ \\
		3 & TOSN & ACM Transactions on Sensor Networks & ACM & http://dblp.uni-trier.de/db/journals/tosn/ \\
		4 & CN & Computer Networks & Elsevier & http://dblp.uni-trier.de/db/journals/cn/ \\
		5 & TCOM & IEEE Transactions on Communications & IEEE & http://dblp.uni-trier.de/db/journals/tcom/ \\
		6 & TWC & IEEE Transactions on Wireless Communications & IEEE & http://dblp.uni-trier.de/db/journals/twc/ \\
		1 & & Ad Hoc Networks & Elsevier & http://dblp.uni-trier.de/db/journals/adhoc/ \\
		2 & CC & Computer Communications & Elsevier & http://dblp.uni-trier.de/db/journals/comcom/ \\
		3 & TNSM & IEEE Transactions on Network and Service Management & IEEE & http://dblp.uni-trier.de/db/journals/tnsm/ \\
		4 & & IET Communications & IET & http://dblp.uni-trier.de/db/journals/iet-com/ \\
		5 & JNCA & Journal of Network and Computer Applications & Elsevier & http://dblp.uni-trier.de/db/journals/jnca/ \\
		6 & MONET & Mobile Networks and Applications & Springer & http://dblp.uni-trier.de/db/journals/monet/ \\
		7 & & Networks & Wiley & http://dblp.uni-trier.de/db/journals/networks/ \\
		8 & PPNA & Peer-to-Peer Networking and Applications & Springer & http://dblp.uni-trier.de/db/journals/ppna/ \\
		9 & WCMC & Wireless Communications and Mobile Computing & Wiley & http://dblp.uni-trier.de/db/journals/wicomm/ \\
		10 & & Wireless Networks & Springer & http://dblp.uni-trier.de/db/journals/winet/ \\
		11 & IOT & IEEE Internet of Things Journal & IEEE & https://dblp.org/db/journals/ iotj/index.html \\
		1 & SIGCOMM & ACM International Conference on Applications, Technologies, Architectures, and Protocols for Computer Communication & ACM & http://dblp.uni-trier.de/db/conf/sigcomm/ index.html \\
		2 & MobiCom & ACM International Conference on Mobile Computing and Networking & ACM & http://dblp.uni-trier.de/db/conf/mobicom/ \\
		3 & INFOCOM & IEEE International Conference on Computer Communications & IEEE & http://dblp.uni-trier.de/db/conf/infocom/ \\
		4 & NSDI & Symposium on Network System Design and Implementation & USENIX & http://dblp.uni-trier.de/db/conf/nsdi/ \\
		1 & SenSys & ACM Conference on Embedded Networked Sensor Systems & ACM & http://dblp.uni-trier.de/db/conf/sensys/ \\
		2 & CoNEXT & ACM International Conference on Emerging Networking Experiments and Technologies & ACM & http://dblp.uni-trier.de/db/conf/conext/ \\
		3 & SECON & IEEE International Conference on Sensing, Communication, and Networking & IEEE & http://dblp.uni-trier.de/db/conf/secon/ \\
		4 & IPSN & International Conference on Information Processing in Sensor Networks & IEEE/ACM & http://dblp.uni-trier.de/db/conf/ipsn/ \\
		5 & MobiSys & ACM International Conference on Mobile Systems, Applications, and Services & ACM & http://dblp.uni-trier.de/db/conf/mobisys/ \\
		6 & ICNP & IEEE International Conference on Network Protocols & IEEE & http://dblp.uni-trier.de/db/conf/icnp/ \\
		7 & MobiHoc & International Symposium on Theory, Algorithmic Foundations, and Protocol Design for Mobile Networks and Mobile Computing & ACM/IEEE & http://dblp.uni-trier.de/db/conf/mobihoc/ \\
		8 & NOSSDAV & International Workshop on Network and Operating System Support for Digital Audio and Video & ACM & http://dblp.uni-trier.de/db/conf/nossdav/ \\
		9 & IWQoS & IEEE/ACM International Workshop on Quality of Service & IEEE & http://dblp.uni-trier.de/db/conf/iwqos/ \\
		10 & IMC & ACM Internet Measurement Conference & ACM/USENIX & http://dblp.uni-trier.de/db/conf/imc/ \\
		
		
	\end{longtable}

	\newpage

	\begin{longtable}{p{2em} p{4.5em}}
		\caption{中国计算机学会部分推荐期刊及会议简表(用于测试跨页表宽低于表题长度的情况)} \label{tab: 中国计算机学会部分推荐期刊及会议简表} \\
		
		\toprule
		\textbf{序号} & \textbf{刊物简称} \\
		\midrule
		\endfirsthead
		
		% 在这里设计首页以外的表题和表头
		\CPcaption{2}{中国计算机学会部分推荐期刊及会议简表(用于测试跨页表宽低于表题长度的情况)}\\
		\toprule
		\textbf{序号} & \textbf{刊物简称} \\
		\midrule
		\endhead
		
		% 在这里设计首页以外的表尾
		\bottomrule
		\multicolumn{2}{r}{续下页} \\  % 如不希望跨页表尾显示任何内容则注释掉即可
		\endfoot
		
		\bottomrule
		\endlastfoot
		
		1 & JSAC \\
		2 & TMC \\
		3 & TON \\
		1 & TOIT \\
		2 & TOMM \\
		3 & TOSN \\
		4 & CN \\
		5 & TCOM \\
		6 & TWC \\
		2 & CC \\
		3 & TNSM \\
		5 & JNCA \\
		6 & MONET \\
		8 & PPNA \\
		9 & WCMC \\
		11 & IOT \\
		1 & SIGCOMM \\
		2 & MobiCom \\
		3 & INFOCOM \\
		4 & NSDI \\
		1 & SenSys \\
		2 & CoNEXT \\
		3 & SECON \\
		4 & IPSN \\
		5 & MobiSys \\
		6 & ICNP \\
		7 & MobiHoc \\
		8 & NOSSDAV \\
		9 & IWQoS \\
		10 & IMC \\
		1 & JSAC \\
		2 & TMC \\
		3 & TON \\
		1 & TOIT \\
		2 & TOMM \\
		3 & TOSN \\
		4 & CN \\
		5 & TCOM \\
		6 & TWC \\
		2 & CC \\
		3 & TNSM \\
		5 & JNCA \\
		6 & MONET \\
		8 & PPNA \\
		9 & WCMC \\
		11 & IOT \\
		1 & SIGCOMM \\
		2 & MobiCom \\
		3 & INFOCOM \\
		4 & NSDI \\
		1 & SenSys \\
		2 & CoNEXT \\
		3 & SECON \\
		4 & IPSN \\
		5 & MobiSys \\
		6 & ICNP \\
		7 & MobiHoc \\
		8 & NOSSDAV \\
		9 & IWQoS \\
		10 & IMC \\
	\end{longtable}

	\clearpage
	\subsection{跨页带附注表格(2025.01.05)}

	很遗憾,\shad{threeparttable}无法做到跨页,而\shad{longtable}又无法像前者那样稳定完美地排版附注。无奈之下,本模板提供了一种繁琐但可行的做法。

	思路是利用\shad{longtable}提供的表尾自定义功能来插入附注,即用户需要修改\shad{longtable}在\shadcmd{endlastfoot}前对表格尾部的设置。为此,模板提供命令:
	
	\shadcmd{tablenotetext(online)(<附注编号悬挂距离>)[<附注上方垂直间隔>]\{<附注总宽度>\}\{<附注标签>\}\{<附注内容>\}}。

	\begin{itemize}
		\item 此命令的第一项可选参数以\shad{()}标识,它仅接受参数\shad{online},作用是将附注编号从默认的上标形式更改为行内形式,即模仿\shad{tablenotes}环境的选项。此间差异可参考表\ref{tab: 跨页带附注表格示例}的附注;
		\item 第二项可选参数仅在设置了第一项可选参数为\shad{online}时有效,用于调整附注编号的悬挂缩进距离,默认是\shad{1em}。如果表格的附注特别多,导致编号过长而与后方文字重叠,那就需要用户手动调整该可选参数,见表\ref{tab: 跨页带附注表格恰巧在附注开始处换页示例};
		\item 第三项可选参数以\shad{[]}标识,在生成的附注与上方内容间的垂直距离不合适时,可通过此可选参数手动对其进行调整,默认为\shad{0bp};
		\item 第四项强制参数对应附注整体的宽度,这个数值需要用户根据表格排版后的宽度反复调整,直至附注与表格尾线等宽为止。我知道这很繁琐,但实在能力有限,想不出更好的办法,自动确定表格的实际宽度真的很难;
		\item 第五项强制参数负责设置附注编号对应的标签,以便后续在表格中使用\shadcmd{tablenoteref\{\}}进行引用,标签同样需要全文唯一;
		\item 第六项强制参数就是附注的实际内容了。
	\end{itemize}

	\textcolor{red}{\textbf{注意:}}使用\shadcmd{tablenotetext}命令的前提是对表格首列进行特定设置,用户\textbf{必须}通过\shad{p{<长度>}}或\shad{m{<长度>}}来人为指定\shad{longtable}的首列宽度,\textcolor{red}{\textbf{切不可}}将之设置为\shad{c}、\shad{r}或\shad{l}。另外,表格中最后一条\shadcmd{tablenotetext}之后不需要\shadcmd{\shadcmd{}},否则表尾与下方文本的间隔将偏大。

	还有一种特殊情况是,跨页表格恰好在附注开始处发生了分页,此时也会产生另类的排版结果。要解决该问题,用户得手动在表格内容的适当位置使用\href{https://mirrors.tuna.tsinghua.edu.cn/CTAN/macros/latex/required/tools/longtable.pdf}{\shad{\color{DarkRed}longtable}}宏包提供的\shadcmd{pagebreak}命令提前断页,代价是前一页底部可能有更多空白,参考表\ref{tab: 跨页带附注表格恰巧在附注开始处换页示例}中的做法。如果你运气尤其差,附注特别长,而且在附注中间跨页了,那我实在无能为力了。

	必须要承认,\shadcmd{tablenotetext}命令很不稳定。其参数在不同的表格中需要特调,甚至在同一表格的不同排版设置下都是如此,可谓一表一参。如遇到不得不用的时候,用户必须要投入很多精力。可惜,这已经是我目前所能做到的极限。


	\newpage

	\begin{longtable}{p{2em} p{4.5em} p{20em} p{6em}}
		\caption{跨页带附注表格示例} \label{tab: 跨页带附注表格示例} \\
		
		\toprule
		\textbf{序号} & \textbf{刊物简称} & \textbf{刊物全称} & \textbf{出版社} \\
		\midrule
		\endfirsthead
		
		% 在这里设计首页以外的表题和表头
		\CPcaption{4}{跨页带附注表格示例}\\
		\toprule
		\textbf{序号} & \textbf{刊物简称} & \textbf{刊物全称} & \textbf{出版社} \\
		\midrule
		\endhead
		
		% 在这里设计首页以外的表尾
		\bottomrule
		\multicolumn{4}{l}{续下页} \\  % 如不希望跨页表尾显示任何内容则注释掉即可
		\endfoot
		
		\bottomrule
		\tablenotetext[-7bp]{37.2em}{tn: 手动跨页表格附注标签IEEE}{IEEE是指电气和电子工程师学会,是一个国际性的专业学会,以促进电气工程、电子工程、计算机科学和相关领域的科学和技术发展为宗旨。成立于1884年,总部位于美国纽约。IEEE 的会员包括来自世界各地的专业人士、工程师、学者和学生,是全球最大的技术专业组织之一。} \\
		\tablenotetext(online)[6bp]{37.2em}{tn: 手动跨页表格附注标签ACM}{ACM是指国际计算机学会,成立于1947年,是一个国际性的科技教育组织,是世界上第一个科学性及教育性计算机学会,总部设在美国纽约。国际计算机学会是世界上最大的计算机领域专业性学术组织,汇集了国际计算机领域教育家,研究人员,工业界人士及学生。ACM致力于提高在中国的活动的规格与影响力。在此基础上,学会成立了ACM中国理事会,为在中国的学会会员与学会活动提供支持。}% 注意这里不需要\\
		\endlastfoot
		
		1 & JSAC & IEEE Journal on Selected Areas in Communications & IEEE \\
		2 & TMC & IEEE Transactions on Mobile Computing & IEEE \\
		3 & TON & IEEE/ACM Transactions on Networking & IEEE/ACM \\
		1 & TOIT & ACM Transactions on Internet Technology & ACM \\
		2 & TOMM & ACM Transactions on Multimedia Computing, Communications and Applications & ACM \\
		3 & TOSN & ACM Transactions on Sensor Networks & ACM \\
		4 & CN & Computer Networks & Elsevier \\
		5 & TCOM & IEEE Transactions on Communications & IEEE \\
		6 & TWC & IEEE Transactions on Wireless Communications & IEEE \\
		1 & & Ad Hoc Networks & Elsevier \\
		2 & CC & Computer Communications & Elsevier \\
		3 & TNSM & IEEE Transactions on Network and Service Management & IEEE \\
		4 & & IET Communications & IET \\
		5 & JNCA & Journal of Network and Computer Applications & Elsevier \\
		6 & MONET & Mobile Networks and Applications & Springer \\
		7 & & Networks & Wiley \\
		8 & PPNA & Peer-to-Peer Networking and Applications & Springer \\
		9 & WCMC & Wireless Communications and Mobile Computing & Wiley \\
		10 & & Wireless Networks & Springer \\
		11 & IOT & IEEE Internet of Things Journal & IEEE \\
		1 & SIGCOMM & ACM International Conference on Applications, Technologies, Architectures, and Protocols for Computer Communication & ACM \\
		2 & MobiCom & ACM International Conference on Mobile Computing and Networking & ACM \\
		3 & INFOCOM & IEEE International Conference on Computer Communications & IEEE \\
		4 & NSDI & Symposium on Network System Design and Implementation & USENIX \\
		1 & SenSys & ACM Conference on Embedded Networked Sensor Systems & ACM \\
		2 & CoNEXT & ACM International Conference on Emerging Networking Experiments and Technologies & ACM \\
		3 & SECON & IEEE International Conference on Sensing, Communication, and Networking & IEEE \\
		4 & IPSN & International Conference on Information Processing in Sensor Networks & IEEE/ACM \\
		5 & MobiSys & ACM International Conference on Mobile Systems, Applications, and Services & ACM \\
		6 & ICNP & IEEE International Conference on Network Protocols & IEEE\tablenoteref{tn: 手动跨页表格附注标签IEEE} \\
		7 & MobiHoc & International Symposium on Theory, Algorithmic Foundations, and Protocol Design for Mobile Networks and Mobile Computing & ACM/IEEE \\
		8 & NOSSDAV & International Workshop on Network and Operating System Support for Digital Audio and Video & ACM\tablenoteref{tn: 手动跨页表格附注标签ACM} \\
		9 & IWQoS & IEEE/ACM International Workshop on Quality of Service & IEEE \\
		10 & IMC & ACM Internet Measurement Conference & ACM/USENIX \\
		
		
	\end{longtable}

	表格后参照文本表格后参照文本表格后参照文本表格后参照文本表格后参照文本表格后参照文本表格后参照文本表格后参照文本表格后参照文本表格后参照文本表格后参照文本表格后参照文本表格后参照文本表格后参照文本表格后参照文本表格后参照文本表格后参照文本表格后参照文本表格后参照文本表格后参照文本表格后参照文本

	\newpage

	\begin{longtable}{m{2em}<{\centering} p{4.5em} p{15em} p{6em}}
		\caption{跨页带附注表格恰巧在附注开始处换页示例} \label{tab: 跨页带附注表格恰巧在附注开始处换页示例} \\
		
		\toprule
		\textbf{序号} & \textbf{刊物简称} & \textbf{刊物全称} & \textbf{出版社} \\
		\midrule
		\endfirsthead
		
		% 在这里设计首页以外的表题和表头
		\CPcaption{4}{跨页带附注表格恰巧在附注中需要换页示例}\\
		\toprule
		\textbf{序号} & \textbf{刊物简称} & \textbf{刊物全称} & \textbf{出版社} \\
		\midrule
		\endhead
		
		% 在这里设计首页以外的表尾
		\bottomrule
		\multicolumn{4}{l}{续下页} \\  % 如不希望跨页表尾显示任何内容则注释掉即可
		\endfoot
		
		\bottomrule
		\tablenotetext[5bp]{32.1em}{tn: 手动跨页表格附注标签Elsevier}{爱思唯尔,创办于1880年,属于RELX集团旗下,总部位于阿姆斯特丹。爱思唯尔是一家荷兰的国际化多媒体出版集团,主要为科学家、研究人员、学生、医学以及信息处理的专业人士提供信息产品和革新性工具。爱思唯尔是全球领先的科学与医学信息服务机构,旗下出版《柳叶刀》《细胞》等2800多种学术期刊。} \setcounter{tablenote}{99}\\
		\tablenotetext(online)(2em)[5bp]{32.1em}{tn: 手动跨页表格附注标签USENIX}{USENIX成立于1975年,当时的名字叫做Unix用户群。它的主要目的是学习及开发Unix以及类似系统。1977年六月,美国电话电报公司的律师告诉用户群他们不能继续使用UNIX这个名字,因为UNIX是美国电话电报公司所拥有的一个商标。所以这个用户群更名成USENIX.从那以后,USENIX逐渐发展成一个倍受尊敬的由计算机操作系统用户,开发者和研究者所组成的机构。}% 注意这里不需要\\
		\endlastfoot
		
		1 & JSAC & IEEE Journal on Selected Areas in Communications & IEEE \\
		2 & TMC & IEEE Transactions on Mobile Computing & IEEE \\
		3 & TON & IEEE/ACM Transactions on Networking & IEEE/ACM \\
		1 & TOIT & ACM Transactions on Internet Technology & ACM \\
		2 & TOMM & ACM Transactions on Multimedia Computing, Communications and Applications & ACM \\
		3 & TOSN & ACM Transactions on Sensor Networks & ACM \\
		4 & CN & Computer Networks & Elsevier \\
		5 & TCOM & IEEE Transactions on Communications & IEEE \\
		6 & TWC & IEEE Transactions on Wireless Communications & IEEE \\
		1 & & Ad Hoc Networks & Elsevier \\
		2 & CC & Computer Communications & Elsevier\tablenoteref{tn: 手动跨页表格附注标签Elsevier} \\
		3 & TNSM & IEEE Transactions on Network and Service Management & IEEE \\
		% 4 & & IET Communications & IET \\
		% 5 & JNCA & Journal of Network and Computer Applications & Elsevier \\
		% 6 & MONET & Mobile Networks and Applications & Springer \\
		% 7 & & Networks & Wiley \\
		% 8 & PPNA & Peer-to-Peer Networking and Applications & Springer \\
		% 9 & WCMC & Wireless Communications and Mobile Computing & Wiley \\
		% 10 & & Wireless Networks & Springer \\
		% 11 & IOT & IEEE Internet of Things Journal & IEEE \\
		1 & SIGCOMM & ACM International Conference on Applications, Technologies, Architectures, and Protocols for Computer Communication & ACM \\
		2 & MobiCom & ACM International Conference on Mobile Computing and Networking & ACM \\
		3 & INFOCOM & IEEE International Conference on Computer Communications & IEEE \\
		4 & NSDI & Symposium on Network System Design and Implementation & USENIX\tablenoteref{tn: 手动跨页表格附注标签USENIX} \\
		1 & SenSys & ACM Conference on Embedded Networked Sensor Systems & ACM \\
		\pagebreak  % 用户可以尝试注释掉这条命令看看现象
		2 & CoNEXT & ACM International Conference on Emerging Networking Experiments and Technologies & ACM \\
		% 3 & SECON & IEEE International Conference on Sensing, Communication, and Networking & IEEE \\
		% 4 & IPSN & International Conference on Information Processing in Sensor Networks & IEEE/ACM \\
		% 5 & MobiSys & ACM International Conference on Mobile Systems, Applications, and Services & ACM \\
		% 6 & ICNP & IEEE International Conference on Network Protocols & IEEE \\
		% 7 & MobiHoc & International Symposium on Theory, Algorithmic Foundations, and Protocol Design for Mobile Networks and Mobile Computing & ACM/IEEE \\
		% 8 & NOSSDAV & International Workshop on Network and Operating System Support for Digital Audio and Video & ACM \\
		% 9 & IWQoS & IEEE/ACM International Workshop on Quality of Service & IEEE \\
		% 10 & IMC & ACM Internet Measurement Conference & ACM/USENIX \\
		
		
		
	\end{longtable}

	表格后参照文本表格后参照文本表格后参照文本表格后参照文本表格后参照文本表格后参照文本表格后参照文本表格后参照文本表格后参照文本表格后参照文本表格后参照文本表格后参照文本表格后参照文本表格后参照文本表格后参照文本表格后参照文本表格后参照文本表格后参照文本表格后参照文本表格后参照文本表格后参照文本
	
	\clearpage
	\section{伪代码}
	
	伪代码基于\href{https://mirrors.sustech.edu.cn/CTAN/macros/latex/contrib/algorithm2e/doc/algorithm2e.pdf}{\shad{\color{DarkRed}algorithm2e}}宏包提供的\shad{algorithm}环境,默认不添加左右侧框线,且顶部框线和底部框线类比规范对表格的要求进行了加粗,字体大小也调整到了\textbf{五号字},与表格保持一致。用户可以在载入文档类时添加\shad{boxruled}选项来恢复左右侧框线。该环境生成的伪码与正文文本保持相同宽度。
	
	除了\href{https://mirrors.sustech.edu.cn/CTAN/macros/latex/contrib/algorithm2e/doc/algorithm2e.pdf}{\shad{\color{DarkRed}algorithm2e}}宏包本身提供的各种条件、循环语句,本模板基于宏包提供的接口,追加了\shad{Do While}和\shad{Loop}循环语句:
	\begin{itemize}
		\item \shadcmd{DoWhile(<紧跟关键字do的文本,可用于添加注释>)\{<循环条件>\}\{<循环体>\}}
		\item \shadcmd{Loop(<紧跟关键字loop的文本,可用于添加注释>)\{<循环体>\}}
	\end{itemize}
	
	
	此外,基于调整后的\shad{algorithm2e}环境,本模板进一步封装了\shad{algo}环境。从名字上可以看出,\shad{algo}环境比\shad{algorithm}环境生成的伪码浮动区域更\textbf{窄}。它除了接受浮动可选参数\shad{[htbp]},还提供了另一可选参数\shad{(<伪码距正文文本边界的总距离>)},该参数控制的是浮动体离正文文本边界的总距离,默认是\shad{4em},即单边缩进\shad{2em},与下方首行文本对齐。两种可选参数可以单独使用或同时使用,但要注意同时使用时的顺序必须与下方例子保持一致:
	
	\begin{verbatim}
		\begin{algo}[<浮动选项>](<伪码距正文文本边界的总距离>)
		    .....
		\end{algo}
	\end{verbatim}
	
	算法\ref{alg: algorithm环境伪码示例}和算法\ref{alg: algo环境伪码示例}分别展示了两种环境默认生成的伪码样式;过程\ref{alg: algorithm环境修改伪码标签示例}和过程\ref{alg: algo环境修改伪码标签并调整宽度示例}展示了如何修改伪码中的一些标签,以及调整\shad{algo}伪码宽度的具体做法。

	
	\begin{algorithm}[!h]
		\caption{algorithm环境伪码示例} \label{alg: algorithm环境伪码示例}
		\Input{1) 输入1;\newline 2) 输入2。}
		\Output{输出结果。}
		伪码行1。
		
		\For(\tcc*[f]{循环条件注释1}){循环条件1}{
			伪码行2。
			
			\tcp{注释2}
			伪码行3。
			
			\DoWhile(\tcc*[f]{循环条件注释3}){循环条件2}{
				伪码行4。
			}
			
			\tcc{loop循环}
			\Loop(\tcc*[f]{注释4}){
				循环体1。
			}
			
			\Repeat(\tcc*[f]{循环条件注释5}){循环条件3}{
				循环体2。
			}
			
			\tcp{if-elseif-else结构示例}
			\uIf(\tcc*[f]{条件注释6}){条件语句5}{
				条件语句5为真,伪码行5。
			}
			\uElseIf(\tcc*[f]{elseif条件语句}){条件语句6}{
				条件语句6为真,伪码行6。
			}
			\Else{
				条件5和6均为假,伪码行7。\tcp*[f]{else代码内容}
			}
			
			\If(\tcc*[f]{条件注释7}){条件语句7}{
				伪码行8。
			}
		}
		\textbf{return} 算法结果。
	\end{algorithm}
	
	\begin{algorithm}[!h]
		\renewcommand{\algorithmcfname}{过程}  % 修改伪码标签需要在\caption{}之前
		\caption{algorithm环境临时修改伪码标签示例} \label{alg: algorithm环境修改伪码标签示例}
		\SetKwInOut{Input}{In}
		\SetKwInOut{Output}{Out}
		\Input{1) 输入1; 2) 输入2。}
		\Output{输出结果。}
		伪码行1。
		
		\For(\tcc*[f]{循环条件注释1}){循环条件1}{
			伪码行2。
			
			\tcp{注释2}
			伪码行3。
			
			\DoWhile(\tcc*[f]{循环条件注释3}){循环条件2}{
				伪码行4。
			}
			
			\tcc{loop循环}
			\Loop(\tcc*[f]{注释4}){
				循环体1。
			}
			
			\Repeat(\tcc*[f]{循环条件注释5}){循环条件3}{
				循环体2。
			}
			\eIf(\tcc*[f]{条件注释6}){条件语句6}{
				为真,伪码行5。
			}{
				条件为假,伪码行6。\tcp*[f]{else代码内容}
			}
			
			\If(\tcc*[f]{条件注释7}){条件语句7}{
				伪码行7。
			}
		}
		\textbf{return} 算法结果。
	\end{algorithm}
	
	\begin{algo}[!h]
		\caption{algo环境伪码示例} \label{alg: algo环境伪码示例}
		\Input{1) 输入1;\newline 2) 输入2。}
		\Output{输出结果。}
		伪码行1。
		
		\For(\tcc*[f]{循环条件注释1}){循环条件1}{
			伪码行2。
			
			\tcp{注释2}
			伪码行3。
			
			\DoWhile(\tcc*[f]{循环条件注释3}){循环条件2}{
				伪码行4。
			}
			
			\tcc{loop循环}
			\Loop(\tcc*[f]{注释4}){
				循环体1。
			}
			
			\Repeat(\tcc*[f]{循环条件注释5}){循环条件3}{
				循环体2。
			}
			\eIf(\tcc*[f]{条件注释6}){条件语句6}{
				为真,伪码行5。
			}{
				条件为假,伪码行6。\tcp*[f]{else代码内容}
			}
			
			\If(\tcc*[f]{条件注释7}){条件语句7}{
				伪码行7。
			}
		}
		\textbf{return} 算法结果。
	\end{algo}
	
	\begin{algo}[!h](8em)
		\renewcommand{\algorithmcfname}{过程}  % 修改伪码标签需要在\caption{}之前
		\caption{algo环境临时修改伪码标签并调整宽度示例} \label{alg: algo环境修改伪码标签并调整宽度示例}
		\SetKwInOut{Input}{In}
		\SetKwInOut{Output}{Out}
		\Input{1) 输入1; 2) 输入2。}
		\Output{输出结果。}
		伪码行1。
		
		\For(\tcc*[f]{循环条件注释1}){循环条件1}{
			伪码行2。
			
			\tcp{注释2}
			伪码行3。
			
			\DoWhile(\tcc*[f]{循环条件注释3}){循环条件2}{
				伪码行4。
			}
			
			\tcc{loop循环}
			\Loop(\tcc*[f]{注释4}){
				循环体1。
			}
			
			\Repeat(\tcc*[f]{循环条件注释5}){循环条件3}{
				循环体2。
			}
			\eIf(\tcc*[f]{条件注释6}){条件语句6}{
				为真,伪码行5。
			}{
				条件为假,伪码行6。\tcp*[f]{else代码内容}
			}
			
			\If(\tcc*[f]{条件注释7}){条件语句7}{
				伪码行7。
			}
		}
		\textbf{return} 算法结果。
	\end{algo}

	\clearpage
	\section{各种列表}

	本模板对\shad{itemize}、\shad{enumerate}和\shad{description}这三种基本列表环境进行了设置,用户可根据实际情况选用。

	\shad{itemize}示例:

	\begin{itemize}
		\item 外层列表条目1。多行填充多行填充多行填充多行填充多行填充多行填充多行填充多行填充多行填充多行填充
		\item 外层列表条目2
		\begin{itemize}
			\item 内层列表条目1。多行填充多行填充多行填充多行填充多行填充多行填充多行填充多行填充多行填充多行填充
			\item 内层列表条目2
		\end{itemize}
		\item 外层列表条目3
	\end{itemize}

	\null

	\shad{enumerate}示例:

	\begin{enumerate}
		\item 外层枚举条目1。多行填充多行填充多行填充多行填充多行填充多行填充多行填充多行填充多行填充多行填充
		\item 外层枚举条目2
		\begin{enumerate}
			\item 内层枚举条目1。多行填充多行填充多行填充多行填充多行填充多行填充多行填充多行填充多行填充多行填充
			\item 内层枚举条目2
		\end{enumerate}
		\item 外层枚举条目3
	\end{enumerate}

	\null

	\shad{description}示例:

	\begin{description}
		\item[描述1] 外层描述条目1。多行填充多行填充多行填充多行填充多行填充多行填充多行填充多行填充多行填充多行填充
		\item[描述2] 外层描述条目2
		\begin{description}
			\item[描述2.1] 内层描述条目1。多行填充多行填充多行填充多行填充多行填充多行填充多行填充多行填充多行填充多行填充
			\item[描述2.2] 内层描述条目2
		\end{description}
		\item[描述3] 外层描述条目3
	\end{description}
	
	\clearpage
	\section{定义、公理、定理、命题、推论、引理、示例、假设、证明}
	
	本模板分别定义了环境:\shad{definition}、\shad{axiom}、\shad{theorem}、\shad{proposition}、\shad{corollary}、\shad{lemma}、\shad{example}、\shad{assumption}和\shad{proof}。示例如下:
	
	% 云南大理段氏嫡传的武功,在点穴功夫中位居天下第一,运功后以右手食指点穴,出指可缓可快,缓时潇洒飘逸,快则疾如闪电,但着指之处,分毫不差。当与敌挣搏凶险之际,用此指法既可贴近径点敌人穴道,也可从远处欺近身去,一中即离,一攻而退,实为克敌保身的无上妙术。

	\begin{definition}[具体名称]
		云南大理段氏嫡传的武功,在点穴功夫中位居天下第一,运功后以右手食指点穴,出指可缓可快,缓时潇洒飘逸,快则疾如闪电。
	\end{definition}

	\begin{axiom}[具体名称]
		云南大理段氏嫡传的武功,在点穴功夫中位居天下第一,运功后以右手食指点穴,出指可缓可快,缓时潇洒飘逸,快则疾如闪电。
	\end{axiom}
	
	\begin{theorem}[具体名称]
		云南大理段氏嫡传的武功,在点穴功夫中位居天下第一,运功后以右手食指点穴,出指可缓可快,缓时潇洒飘逸,快则疾如闪电。

		\begin{enumerate}
			\item 当与敌挣搏凶险之际,用此指法既可贴近径点敌人穴道,也可从远处欺近身去,一中即离,一攻而退,实为克敌保身的无上妙术。
			\begin{enumerate}
				\item 当与敌挣搏凶险之际,用此指法既可贴近径点敌人穴道,也可从远处欺近身去,一中即离,一攻而退,实为克敌保身的无上妙术。
			\end{enumerate}
		\end{enumerate}
	\end{theorem}
	
	\begin{proposition}[具体名称]
		云南大理段氏嫡传的武功,在点穴功夫中位居天下第一,运功后以右手食指点穴,出指可缓可快,缓时潇洒飘逸,快则疾如闪电。
	\end{proposition}
	
	\begin{corollary}[具体名称]
		云南大理段氏嫡传的武功,在点穴功夫中位居天下第一,运功后以右手食指点穴,出指可缓可快,缓时潇洒飘逸,快则疾如闪电。
	\end{corollary}
	
	\begin{lemma}[具体名称]
		云南大理段氏嫡传的武功,在点穴功夫中位居天下第一,运功后以右手食指点穴,出指可缓可快,缓时潇洒飘逸,快则疾如闪电。
	\end{lemma}

	\begin{example}[具体名称]
		云南大理段氏嫡传的武功,在点穴功夫中位居天下第一,运功后以右手食指点穴,出指可缓可快,缓时潇洒飘逸,快则疾如闪电。
	\end{example}

	\begin{assumption}[具体名称]
		云南大理段氏嫡传的武功,在点穴功夫中位居天下第一,运功后以右手食指点穴,出指可缓可快,缓时潇洒飘逸,快则疾如闪电。
	\end{assumption}
	
	\begin{proof}
		云南大理段氏嫡传的武功,在点穴功夫中位居天下第一,运功后以右手食指点穴,出指可缓可快,缓时潇洒飘逸,快则疾如闪电。
		\begin{itemize}
			\item 当与敌挣搏凶险之际,用此指法既可贴近径点敌人穴道,也可从远处欺近身去,一中即离,一攻而退,实为克敌保身的无上妙术。
			\begin{itemize}
				\item 当与敌挣搏凶险之际,用此指法既可贴近径点敌人穴道,也可从远处欺近身去,一中即离,一攻而退,实为克敌保身的无上妙术。
			\end{itemize}
		\end{itemize}
	\end{proof}
	
	\newpage

	\section{脚注}
	
	本模板使用包含了带圈数字的字体来替换LaTeX绘制的带圈数字,提供了充足的带圈编号数量,同时保证了带圈脚注编号足够优雅。
	
	在正文中加入脚注直接在需要放置脚注标签的位置使用\shadcmd{footnote\{<脚注内容>\}}即可。
	
	在其他环境中,如表格,则需要需要使用\shadcmd{footnotemark}配合\shadcmd{footnotetext\{<脚注文本>\}}。在需要放置脚注标签的位置使用\shadcmd{footnotemark},然后在环境外使用\shadcmd{footnotetext\{<脚注文本>\}}指明脚注内容\footnote{更详细的使用方法参考\href{https://blog.csdn.net/xovee/article/details/127563209}{\shad{\color{DarkRed}LaTeX脚注}}。冗余文本用于展示脚注内容发生换行后的情况;冗余文本用于展示脚注内容发生换行后的情况;冗余文本用于展示脚注内容发生换行后的情况。}。
	
	
	\section{模板中的各种编号}
	
	标题、图片、表格、伪码、公式、定义、定理、命题、推论、引理、证明、脚注这些文档元素的编号都是自行计算并生成的,无需劳烦用户。\textbf{但形如(1-1a)的子公式编号不能完全自动生成},为此,模板提供了较为便捷的\shadcmd{subeqtag[<子公式编号标签>]}命令让用户花尽可能少的精力做到这一点,且保证完全不会出错。
	
	该需求往往出现在数学模型的约束中,最常用的方式可能是使用\shadcmd{tag\{\}}命令显示指定某条约束的编号。但是,该方式操作繁琐,而且在后续需要调整约束顺序或增删约束时很容易漏改某些tag导致子公式编号混乱,对论文作者来说这是很容易被忽略的问题。
	
	本模板提供的\shadcmd{subeqtag[<子公式编号标签>]}命令完全避免了上述问题。您只需要在对应的约束后使用\shadcmd{subeqtag},该约束就会被赋予与当前主公式编号保持一致的下级编号。并且,对连续的多个约束使用该命令会\textbf{自动生成}递增的子公式编号,交换约束顺序编号也会自行更正,断不可能出错。
	
	如果您需要在正文中引用某个子公式编号,那么可以像往常一样在\shadcmd{subeqtag}之后使用\shadcmd{label\{<编号标签>\}},或者直接指定\shadcmd{subeqtag[<子公式编号标签>]}的可选参数,一条命令就搞定,非常人性化。
	
	下面的源码将产生式\eqref{eq: obj 1}\textasciitilde \eqref{eq: constriant gamma}对应的例子,其中式\eqref{eq: constraint x}和\eqref{eq: constriant gamma}使用了\shadcmd{subeqtag[<子公式编号标签>]}的可选参数。
	
	
	\begin{verbatim}
		\begin{align}
			\max \log \left(x^2 + y^2 + z^2 + v^2 + g^2 + m^2 + k^2\right) \\
			\text{s.t.} \quad x \leq 1, \subeqtag[eq: constraint x] \\
			y \leq 2, \subeqtag \\
			z \leq 4, \subeqtag
		\end{align}
		
		\begin{align}
			\min \left(\boldsymbol{\alpha} + \beta + \gamma_{中文}\right)^2  \\
			\text{s.t.} \quad \boldsymbol{\alpha} \leq 9, \subeqtag \\
			\beta \geq -10, \subeqtag \\
			\gamma_{中文} \geq 8, \subeqtag[eq: constriant gamma]
		\end{align}
	\end{verbatim}
	
	\begin{align}
		\thinmuskip=-3mu \medmuskip=-2mu \thickmuskip=-1mu
		\max \log \left(x^2 + y^2 + z^2 + v^2 + g^2 + m^2 + k^2\right) \\
		\text{s.t.} \quad x \leq 1, \subeqtag[eq: constraint x] \\
		y \leq 2, \subeqtag \\
		z \leq 4, \subeqtag
	\end{align}
	
	\begin{align}
		\min \left(\boldsymbol{\alpha} + \beta + \gamma_{中文}\right)^2 \label{eq: obj 2} \\
		\text{s.t.} \quad \boldsymbol{\alpha} \leq 9, \subeqtag \\
		\beta \geq -10, \subeqtag \\
		\gamma_{中文} \geq 8, \subeqtag[eq: constriant gamma]
	\end{align}
	

	$\boldsymbol{x}^2 + \mathrm{y}^2 + z^2 + \mathcal{B}^2 + \mathcal{M}^2 + \mathbb{I}^2 + v^2 + g^2 + m^2 + k^2$
	
	$\check{\boldsymbol{C}}^t_n + \hat{p}^j_{n, s} + \tilde{r}^t_{u,b} + \overline{i}^t_{x, y} + \acute{\alpha}^t_v + \acute{\boldsymbol{\alpha}}^t + \hat{\boldsymbol{\alpha}}^t + f^t_j + f^{"t}_j + l^t_j + l^{"t}_j$

	$\xlongequal{uvw}, \cdot, \cdots, \xLeftrightarrow{xyz}, \mathcal{N} \triangleq \left\{1, \dots, N\right\}, \times, \partial, \emptyset, \in, \subseteq, \leftarrow, \rightarrow, \leq, \geq$
	
	有两点需要提醒,\shadcmd{subeqtag[<子公式编号标签>]}的可选参数全文不可重复定义,因为它本质上还是调用的\shadcmd{label\{<编号标签>\}},不用多说。另外,尽管使用\shadcmd{subeqtag[<子公式编号标签>]}的可选参数指定的标签本质上是基于\shadcmd{label\{<编号标签>\}}进行的封装,在TexStudio这样的编辑器上使用\shadcmd{ref\{<编号标签>\}}或\shadcmd{eqref\{<编号标签>\}}却不会自动弹出这些标签的选项,需要用户手动输入;而如果是直接用\shadcmd{label\{<编号标签>\}}指定的标签,引用时会出现在提醒选项中,用户可以直接选择,这算是\shadcmd{subeqtag[<子公式编号标签>]}不太方便的点,可惜我并不知道该如何解决。
	
	
	\section{在标题中排版数学符号\texorpdfstring{$\tilde{r}^t_{u,b}, \acute{\alpha}^t_v, \check{\boldsymbol{C}}^t_n$}{示例}}

	尽管我不建议各位在标题中排版数学符号(因为规范甚至不建议在标题中排版英文缩略词),但如果你非排版不可,那可参考本节标题的做法,使用\href{https://mirrors.tuna.tsinghua.edu.cn/CTAN/macros/latex/contrib/hyperref/doc/hyperref-doc.pdf}{\shad{\color{DarkRed}hyperref}}宏包(模板已载入该宏包)提供的\shadcmd{texorpdfstring\{<TeXstring>\}\{<PDFstring>\}}命令,该命令的具体用法参考这个帖子:\href{https://blog.csdn.net/qq_42679415/article/details/139592054}{\shad{\color{DarkRed}texorpdfstring使用方法}}。

	
	\section{引用}
	
	对公式、图片、表格、伪码、定义、定理、命题、推论、引理、证明等编号的引用直接用\shadcmd{ref\{<编号label>\}}即可,其中需要带括号公式编号则使用\shadcmd{eqref\{<公式label>\}}。
	
	若要对子图题编号进行完整引用直接使用\shadcmd{ref\{<子图题标签>\}}即可,\shad{DissertUESTC}文档类默认生成形如\shad{1-1(a)}的完整编号,但若用户指定了文档类的\shad{subfigsimple}选项,则会生成形如\shad{1-1a}的完整编号(\textit{注:学位论文撰写规范中并未明确说明引用子图编号应该采用哪种形式,但我翻了本中文专著,里面采用的\shad{1-1(a)}形式,故而设为了默认样式});反之,若只希望单独引用子图题编号,比如在图题结尾按编号添加子图题文本,则需要使用\shadcmd{subref\{<子图题标签>\}},它将生成形如\shad{(a)}的单独编号。
	
	对参考文献的行内引用直接使用\shadcmd{cite\{<参考文献label>\}},以上标形式引用则使用\shadcmd{citess\{<参考文献label>\}}。
	
	参考文献的引用是基于\shad{natbib}宏包实现的,单次引用多篇参考文献时会自动排序并压缩序号(如果可以的话)。

	另外,研究生论文规范要求正文中引用的公式编号样式采用英文括号,即\shad{(1-1)};而本科论文规范中则要求是中文括号,即\shad{(1-1)}。在公式右侧的编号中,两者均采用英文括号。此间区别完全由相应的选项(\shad{doctor}/\shad{prodoctor}/\shad{master}/\shad{promaster}/ \shad{bachelor})控制,用户无需过问,但一定要确保自己写对了选项。
	
	
	\section{参考文献编译}
	
	本模板实现了规范中列举的\textbf{期刊论文}、\textbf{会议论文}、\textbf{专著}、\textbf{学位论文}、\textbf{报纸文章}、\textbf{报告}、\textbf{授权专利}、\textbf{标准}、\textbf{电子文献},共计9种文献类型的排版风格。
	
	本模板为这些文献类型定义的\shad{.bib}数据库条目\textbf{类型标识}分别为\shad{article}、\shad{inproceedings/conference}、\shad{book}、\shad{mastersthesis/phdthesis}、\shad{news}、\shad{report}、\shad{patent}、\shad{standard}、\shad{digital}。
	
	不同文档类型条目包含不同的域,下面列举了一些\href{https://gr.uestc.edu.cn/xiazai/114/3917}{研究生学位论文撰写规范}中用作示例的参考文献对应的\shad{.bib}数据库形式,完全覆盖上述9种文献类型:
	
	\begin{verbatim}
	@book{教育部国家语言文字工作委员2018,
	    author={教育部国家语言文字工作委员},
	    title={通用规范汉字},
	    address={北京},
	    publisher={语文出版社},
	    year={2018},
	    language={schinese},
	}
	
	@standard{学位论文编写规范555,
	    author={全国信息与文献标准化技术委员},
	    title={学位论文编写规范},
	    number={GB/T 7713.1-2006},
	    address={北京},
	    publisher={中国标准出版社},
	    year={2007},
	    pages={17-20},
	}
	
	@article{王晓琰2019关于连续出版会议论文著录格式的探讨,
	    title={关于连续出版会议论文著录格式的探讨},
	    author={王晓琰 and 殷建芳 and 王晓峰 and 邓迎 and 杨蕾},
	    journal={学报编辑丛论},
	    number={0},
	    year={2019},
	    pages={162-165},
	    language={schinese},
	}
	
	@article{hu2014domain,
	    title={Domain decomposition method based on integral equation for solution of
	    scattering from very thin, conducting cavity},
	    author={Hu, Jun and Zhao, Ran and Tian, Mi and Zhao, Huapeng and Jiang,
	    Ming and Wei, Xiang and Nie, Zai Ping},
	    journal={IEEE Transactions on Antennas and Propagation},
	    volume={62},
	    number={10},
	    pages={5344-5348},
	    year={2014},
	    publisher={IEEE}
	}
	
	@inproceedings{bergamasco2015adopting,
	    title={Adopting an unconstrained ray model in light-field cameras for 3d
	    shape reconstruction},
	    author={Bergamasco, Filippo and Albarelli, Andrea and Cosmo, Luca and Torsello,
	    Andrea and Rodola, Emanuele and Cremers, Daniel},
	    booktitle={IEEE Conference on Computer Vision and Pattern Recognition},
	    pages={3003-3012},
	    year={2015},
	    organization={Boston, USA}
	}
	
	@article{xue2024survey,
	    title={A survey of beam management for mmWave and THz communications
	    towards 6G},
	    author={Xue, Qing and Ji, Chengwang and Ma, Shaodan and Guo, Jiajia and Xu,
	    Yongjun and Chen, Qianbin and Zhang, Wei},
	    journal={IEEE Communications Surveys \& Tutorials},
	    year={2024},
	    pages={1-41},
	    publisher={IEEE}
	}
	
	@book{罗杰斯2011,
	    author={罗杰斯},
	    title={西方文明史:问题与源头},
	    translator={潘惠霞 and 魏婧 and 杨艳 and 汤玲},
	    edition={2},
	    address={大连},
	    publisher={东北财经大学出版社},
	    year={2011},
	    pages={1-353},
	    language={schinese},
	}
	
	@book{harrington1993field,
	    title={Field computation by moment methods},
	    author={Harrington, Roger F},
	    year={1993},
	    pages={76-112},
	    edition={3},
	    address={New York},
	    publisher={Wiley-IEEE Press}
	}
	
	@digital{电子文献1,
	    author={Deverell, W and gler, D},
	    title={A companion to California history},
	    type={M/OL},
	    modifydate={2013-11-15},
	    url={http://onlinelibrary.wiley.com/doi/.ch2/summary},
	    doi={10.1002/9781444305036},
	    address={New York},
	    publisher={John Wiley \& Sons},
	    year={2013},
	    pages={21-22},
	    citedate={2014-06-24},
	}
	
	@digital{电子文献2,
	    author={Clerc, M},
	    title={Discrete particle swarm optimization: a fuzzy combinatorial box},
	    type={EB/OL},
	    modifydate={2010-07-16},
	    url={http://clere.maurice.free.fr/pso/Fuzzy_Discrere_PSO/Fuzzy_DPSO.html},
	}
	
	@mastersthesis{陈念永2001毫米波细胞生物效应及抗肿瘤研究,
	    author={陈念永},
	    title={毫米波细胞生物效应及抗肿瘤研究},
	    address={成都},
	    school={电子科技大学},
	    year={2001},
	    pages={50-60},
	}
	
	@news{顾春20122,
	    author={顾春},
	    title={牢牢把握稳中求进的总基调},
	    publisher={人民日报},
	    year={2012},
	    month={03},
	    day={31},
	    number={3},
	}
	
	@report{冯西桥1997,
	    author={冯西桥},
	    title={核反应堆压力容器的{LBB}分析},
	    address={北京},
	    publisher={清华大学核能技术设计研究院},
	    year={1997},
	}
	
	@patent{肖珍新2012,
	    author={肖珍新},
	    title={一种新型排渣阀调节降温装置},
	    number={ZL201120085830.0},
	    year={2012},
	    month={04},
	    day={25},
	}

	@phdthesis{陈念永2001毫米波细胞生物效应及抗肿瘤研究无页码,
	    author={陈念永},
	    title={毫米波细胞生物效应及抗肿瘤研究(无页码测试)},
	    address={成都},
	    school={电子科技大学},
	    year={2001},
	}
	\end{verbatim}
	
	这些\shad{.bib}数据依次编译后的结果见本文档中附上的参考文献列表,用户可对应查看。感兴趣的朋友可与\href{https://gr.uestc.edu.cn/xiazai/114/3917}{\shad{\color{DarkRed}研究生学位论文撰写规范}}中给出的结果进行对比,看看是否做到了完全复刻。

	当用户漏掉了参考文献需要的强制域时,BibTeX编译会报错。在VScode中,编译将直接中断;在TeXstudio中,编译不会中断,但log窗口会打印错误信息。这是一种规范控制手段,并非模板bug。遇到这类问题,用户应该自行筛查漏掉了哪些信息。
	
	生成参考文献最耗费精力的是维护正确的\shad{ref.bib}数据库。在这之后,只需要在正文的对应位置使用以下两行代码即可插入完整的参考文献列表:
	\begin{verbatim}
		\bibliographystyle{DissertUESTC}
		\bibliography{ref}
	\end{verbatim}
	
	多说两句:
	\begin{itemize}
		\item 对于某些缺少非必要信息的文献,本模板提供的\shad{.bst}文件依然可以正确处理。比如\cite{王晓琰2019关于连续出版会议论文著录格式的探讨}这篇期刊论文缺少卷号,它仍能仅排版期号,这是符合规范的。再比如,文献\cite{电子文献2}比文献\cite{电子文献1}少了\textbf{出版地}、\textbf{出版者}等信息,依然能正常排版;但是注意,\cite{电子文献2}已经是这类文献的最简形式,不可再缺信息。
		
		\item 对中文参考文献,如果希望将它们的第四顺位及以后的作者显示为\shad{“等”},则必须要在它们的bib条目中加入\shad{language=\{\}}域,并将值设置为\shad{schinese}。这是文献编译引擎判断该条参考文献是否是中文的唯一依据。类似的,\cite{罗杰斯2011}中的\shad{“等译”}、\shad{“2版”}均靠设置\shad{language=\{schinese\}}实现。我的建议是,虽然\shad{language}域并非是强制添加的,但对于中文文献,最好将其添加进去。
		
		\item 对电子文献,其类型众多,因此需要用户通过\shad{type=\{\}}域显式指定,如文献\cite{电子文献1}和\cite{电子文献2};而对其他的文献类型,只要在\shad{@}符号后输入了正确的类型标识,对应的类型标签会自动生成,无需用户手动逐条添加。
		
		\item (2025.01.02)需要特别说明的参考文献类型是\shad{mastersthesis/phdthesis},即学位论文。学校在撰写规范中提到参考文献排版应该遵循国标\href{https://lib.tsinghua.edu.cn/wj/GBT7714-2015.pdf}{\color{DarkRed}GB/T 7714-2015},在此国标中,对学位论文的引用不需要页码信息。但是,\textbf{学校的规范中又明确为学位论文引用添加了页码信息}。为此,我询问了学位办的相关老师,得到的答复大致是:“\textit{我翻阅了手边信通学院学生的论文,他们是写了页码信息的。但这个要求并没有那么严格,不会说没写就评审不过,历年也没有出现因为这个不过的情况,评审老师也没有严格挑。学位论文的质量并不是通过这个来评判的,不过我们这边还是建议写上}”。
   
		本模板原本的实现方式是遵循学校的撰写规范,将学位论文的页码信息设置为了强制域。如果缺少该信息,使用VScode作为编辑器时将报错而无法完成编译;TeXstudio则能跳过这种小问题继续编译,但BibTeX仍会输出错误提醒。GitHub用户\href{https://github.com/zealrussell}{\color{DarkRed}@zealrussell}在使用VScode撰写论文时,发现不为学位论文这类参考文献添加页码信息将无法顺利编译,遂发起了issue。
		
		经过查阅相关文件,以及向学位办老师求证,本模板调整了此类文献的排版规则。现在,学位论文的`pages`域将不再是强制域,缺少该信息不会再中断编译过程,但会输出警告,提醒用户某条参考文献条目缺少页码信息,见参考文献\cite{陈念永2001毫米波细胞生物效应及抗肿瘤研究无页码}。如果你使用TeXstudio,则在编译参考文献辅助文件时BibTeX会发出该警告(图\ref{fig: TeXstudio对参考文献缺失信息的提醒});如果你使用VScode,则需要去检查\shad{problems}窗口输出的信息。它是按文件对警告进行分类的,你需要先定位到\shad{ref.bib}(图\ref{fig: VScode对参考文献缺失信息的提醒})。本模板将是否在引用学位论文时添加页码信息的选择权交予用户,但我个人仍建议各位遵循学校的规范。
	\end{itemize}
	
	\begin{figure}[!h]
		\centering
		\includegraphics[width=0.9\linewidth]{TeXstudio_bibtex}
		\caption{TeXstudio对参考文献缺失可选域的提醒} \label{fig: TeXstudio对参考文献缺失信息的提醒}
	\end{figure}
	\begin{figure}[!h]
		\centering
		\includegraphics[width=0.9\linewidth]{VScode_bibtex}
		\caption{VScode对参考文献缺失可选域的提醒} \label{fig: VScode对参考文献缺失信息的提醒}
	\end{figure}
	
	\acknowledgement
	
	% \null\newpage

	致谢内容

	
	%% 参考文献部分
%	\nocite{*}
	\bibliographystyle{DissertUESTC}
	\bibliography{ref}
	
	% 附录起始位置
	\appendix
	
	\chapter{九阴真经原本}
	\section{气沉丹田}
	\newpage
	\section{多页测试}
	
	\chapter{黯然销魂掌秘籍}
	\section{真气运转}
	\newpage
	\section{多页测试}
	
	\achievement % 仅研究生用
	
	\section*{发表论文:}
	
	\begin{enumerate}
	    \item \textbf{作者1}, 作者2*, 作者3, 作者4. Domain decomposition method based on integral equation for solution of scattering from very thin, conducting cavity. \emph{IEEE Transactions on Antennas and Propagation}, 2014, 62(10): 5344--5348. (\textbf{CCF评级}, \underline{中科院分区}, IF: 98.8)
	    
		\setcounter{enumi}{98}
	    
		\item \textbf{作者1}, 作者2*, 作者3, 作者4. Domain decomposition method based on integral equation for solution of scattering from very thin, conducting cavity. \emph{IEEE Transactions on Antennas and Propagation}, 2014, 62(10): 5344--5348. (\textbf{CCF评级}, \underline{中科院分区}, IF: 98.8)
	\end{enumerate}
	
	\section*{发明专利:}
	

	\begin{enumerate}
		
		\item \textbf{作者1}, 作者2*, 作者3, 作者4。一种基于xxxxx的真气运转方法: ZL201120846830.0. 2023--02--20.
		
	\end{enumerate}
	
	\section*{参与项目:}
	
	\begin{itemize}
		\item 项目号. 项目名称. 项目级别, 2020.01--2022.12.
	\end{itemize}

	\newpage
	\section*{多页测试}

	%%%%%% 开启“外文资料原文”
	%% \originalliterature{<外文标题>}{<外文作者>}
	\originalliterature{How to Take Revenge When My Father's
	Murderer is Suspected to Be a Famous Hero}{Yang Guo}

	%% 这部分内容务必使用模板提供的首字母大写的标题命令生成对应的外文原文标题,
	%% 否则,对应内容将出现在正文的目录和pdf的书签中,非常冗余。
	%% 另外,可使用的各级标题有\Section{}、\Subsection{}、\Subsubsection{},没有chapter

	\textit{Abstract}——When Yang was a teenager, his mother contracted a disease and died, and he then lived a wandering life. When he met Guo Jing and his wife, they took care of him. However, due to the conflict between Yang and Guo Fu, Guo Jing sent him to learn martial arts in the Quanzhen Sect.

	\textit{Index Terms}——Martial arts, apostasy, revenge, fighting against the enemy, martial arts, apostasy, revenge, fighting against the enemy

	\Section{Introduction}

	\begin{table}[htp]
		\captionsetup{list=no}% 阻止此表格显示在表目录中
		\caption{Example of a table}
		\begin{tabular}{ccccc}
			\toprule
			\multirow{2}{*}{Column0} &  \multicolumn{2}{c}{Column1\tnote{1}} & \multicolumn{2}{c}{Column2\tnote{2}} \\
			\cmidrule(lr){2-3}\cmidrule(l){4-5}
			~     & subcolumn1 & subcolumn2 & subcolumn1 & subcolumn2 \\
			\midrule
			Row1  & element11 & element12 &element13 & element14 \\
			Row2  & element21 & element22 &element23 & element24 \\
			\cmidrule{2-3}\cmidrule{4-5}
			Row3  & element31 & element32 &element33 & element34 \\
			\bottomrule
		\end{tabular}
	\end{table}


	\Subsection{Contributions}


	\Subsubsection{While while while}

	\begin{longtable}{p{2em} p{6em}}
		\captionsetup{list=no}% 阻止此表格显示在表目录中
		\caption{Recommended journals and conferences}\\
		
		\toprule
		\textbf{Num} & \textbf{Abbreviation} \\
		\midrule
		\endfirsthead
		
		% 在这里设计首页以外的表题和表头
		\CPcaption{2}{Recommended journals and conferences}\\
		\toprule
		\textbf{Num} & \textbf{Abbreviation} \\
		\midrule
		\endhead
		
		% 在这里设计首页以外的表尾
		\bottomrule
		\multicolumn{2}{l}{to next page} \\  % 如不希望跨页表尾显示任何内容则注释掉即可
		\endfoot
		
		\bottomrule
		\endlastfoot
		
		1 & JSAC \\
		2 & TMC \\
		3 & TON \\
		1 & TOIT \\
		2 & TOMM \\
		3 & TOSN \\
		4 & CN \\
		5 & TCOM \\
		6 & TWC \\
		2 & CC \\
		3 & TNSM \\
		5 & JNCA \\
		6 & MONET \\
		8 & PPNA \\
		9 & WCMC \\
		11 & IOT \\
		% 1 & SIGCOMM \\
		% 2 & MobiCom \\
		% 3 & INFOCOM \\
		% 4 & NSDI \\
		% 1 & SenSys \\
		% 2 & CoNEXT \\
		% 3 & SECON \\
		% 4 & IPSN \\
		% 5 & MobiSys \\
		% 6 & ICNP \\
		% 7 & MobiHoc \\
		% 8 & NOSSDAV \\
		% 9 & IWQoS \\
		% 10 & IMC \\
		% 1 & JSAC \\
		% 2 & TMC \\
		% 3 & TON \\
		% 1 & TOIT \\
		% 2 & TOMM \\
		% 3 & TOSN \\
		% 4 & CN \\
		% 5 & TCOM \\
		% 6 & TWC \\
		% 2 & CC \\
		% 3 & TNSM \\
		% 5 & JNCA \\
		% 6 & MONET \\
		% 8 & PPNA \\
		% 9 & WCMC \\
		% 11 & IOT \\
		% 1 & SIGCOMM \\
		% 2 & MobiCom \\
		% 3 & INFOCOM \\
		% 4 & NSDI \\
		% 1 & SenSys \\
		% 2 & CoNEXT \\
		% 3 & SECON \\
		% 4 & IPSN \\
		% 5 & MobiSys \\
		% 6 & ICNP \\
		% 7 & MobiHoc \\
		% 8 & NOSSDAV \\
		% 9 & IWQoS \\
		% 10 & IMC \\
	\end{longtable}
	
	\Section{Related works}

	\begin{equation}
		x^2 + y^2 + z^2
	\end{equation}
	
	\Section{System model}

	%% 在“外文资料原文”部分,排版定义、公理、定理、命题、推论、引理、示例、假设需使用对应首字母大写的环境,如下所示

	\begin{Definition}[Name]
		Later, he escaped from the Quanzhen Sect and met Xiao Longnian at the ancient tomb, where he practised kung fu with Xiao Longnian.
	\end{Definition}

	\begin{Axiom}[Name]
		Later, he escaped from the Quanzhen Sect and met Xiao Longnian at the ancient tomb, where he practised kung fu with Xiao Longnian.
	\end{Axiom}
	
	\begin{Theorem}[Name]
		Later, he escaped from the Quanzhen Sect and met Xiao Longnian at the ancient tomb, where he practised kung fu with Xiao Longnian.
	\end{Theorem}
	
	\begin{Proposition}[Name]
		Later, he escaped from the Quanzhen Sect and met Xiao Longnian at the ancient tomb, where he practised kung fu with Xiao Longnian.
	\end{Proposition}
	
	\begin{Corollary}[Name]
		Later, he escaped from the Quanzhen Sect and met Xiao Longnian at the ancient tomb, where he practised kung fu with Xiao Longnian.
	\end{Corollary}
	
	\begin{Lemma}[Name]
		Later, he escaped from the Quanzhen Sect and met Xiao Longnian at the ancient tomb, where he practised kung fu with Xiao Longnian.
	\end{Lemma}

	\begin{Example}[Name]
		Later, he escaped from the Quanzhen Sect and met Xiao Longnian at the ancient tomb, where he practised kung fu with Xiao Longnian.
	\end{Example}

	\begin{Assumption}[Name]
		Later, he escaped from the Quanzhen Sect and met Xiao Longnian at the ancient tomb, where he practised kung fu with Xiao Longnian.
	\end{Assumption}
	
	\begin{proof}
		Later, he escaped from the Quanzhen Sect and met Xiao Longnian at the ancient tomb, where he practised kung fu with Xiao Longnian.
	\end{proof}
	
	\Section{Algorithm design}

	\begin{algo}[!h](8em)
		\caption{Example of an algorithm}
		\Input{1) input1; 2) input2.}
		\Output{result.}

		Pseudocode line 1.
		
		\For(\tcc*[f]{for note 1}){Condition 1}{
			Pseudocode line 2.
			
			\tcp{note 2}
			Pseudocode line 3.
			
			\DoWhile(\tcc*[f]{while note 3}){Condition 2}{
				Pseudocode line 4.
			}
			
			\tcc{loop cycle}
			\Loop(\tcc*[f]{note 4}){
				cycle body 1.
			}
			
			\Repeat(\tcc*[f]{repeat note 5}){Condition 3}{
				cycle body 2.
			}
			\eIf(\tcc*[f]{if note 6}){Condition 6}{
				when true,Pseudocode line 5.
			}{
				when false,Pseudocode line 6.\tcp*[f]{repeat cycle}
			}
		}
		\textbf{return} result.
	\end{algo}
	
	\Section{Experiments}

	Example of itemize:

	\begin{itemize}
		\item When Yang was a teenager, his mother contracted a disease and died, and he then lived a wandering life.
		\begin{itemize}
			\item When Yang was a teenager, his mother contracted a disease and died, and he then lived a wandering life.
			\item When Yang was a teenager, his mother contracted a disease and died, and he then lived a wandering life.
		\end{itemize}
		\item When Yang was a teenager, his mother contracted a disease and died, and he then lived a wandering life.
	\end{itemize}

	\null

	Example of enumerate:

	\begin{enumerate}
		\item When Yang was a teenager, his mother contracted a disease and died, and he then lived a wandering life.
		\begin{enumerate}
			\item When Yang was a teenager, his mother contracted a disease and died, and he then lived a wandering life.
			\item When Yang was a teenager, his mother contracted a disease and died, and he then lived a wandering life.
		\end{enumerate}
		\item When Yang was a teenager, his mother contracted a disease and died, and he then lived a wandering life.
	\end{enumerate}
	
	\Section{Conclusions}

	\begin{figure}[!htb]
		\centering
		\includegraphics[width=0.5\linewidth]{洪七公2}
		\captionsetup{list=no}% 阻止此图片显示在图目录中
		\caption{Dare you touch it? Little devil}
	\end{figure}
	
	\begin{figure}[!htb]
		\centering
		\includegraphics[width=0.5\linewidth]{杨过3}
		\captionsetup{list=no}% 阻止此图片显示在图目录中
		\caption{Hard to hold}
	\end{figure}


	%%%%%% 开启“外文资料译文”
	%% \translatedliterature{<译文标题>}{<原文作者>}
	\translatedliterature{关于我的杀父仇人疑似是名震天下的
	大侠时该如何报仇}{杨过}

	%% 这部分内容务必使用模板提供的首字母大写的标题命令生成对应的外文译文标题,
	%% 否则,对应内容将出现在正文的目录和pdf的书签中,非常冗余。
	%% 另外,可使用的各级标题有\Section{}、\Subsection{}、\Subsubsection{},没有chapter

	\textit{摘要}——杨过少年时期母亲染病而亡,随后他便过着四处流浪的生活。后来遇到郭靖夫妇,便由他们照看。但之后因杨过与郭芙等人之间的矛盾,郭靖便送其去全真派习武。

	\textit{关键词}——练武,离经叛道,复仇,抗敌,练武,离经叛道,复仇,抗敌,练武,离经叛道,复仇,抗敌,练武,离经叛道,复仇,抗敌

	\Section{引言}

	\begin{table}[htbp]
		\captionsetup{list=no}% 阻止此表格显示在表目录中
		\caption{表格样例}
		\begin{tabular}{ccccc}
			\toprule
			\multirow{2}{*}{Column0} &  \multicolumn{2}{c}{Column1\tnote{1}} & \multicolumn{2}{c}{Column2\tnote{2}} \\
			\cmidrule(lr){2-3}\cmidrule(l){4-5}
			~     & subcolumn1 & subcolumn2 & subcolumn1 & subcolumn2 \\
			\midrule
			Row1  & element11 & element12 &element13 & element14 \\
			Row2  & element21 & element22 &element23 & element24 \\
			\cmidrule{2-3}\cmidrule{4-5}
			Row3  & element31 & element32 &element33 & element34 \\
			\bottomrule
		\end{tabular}
	\end{table}


	\Subsection{贡献}


	\Subsubsection{好好好}

	\begin{longtable}{p{2em} p{4.5em}}
		\captionsetup{list=no}% 阻止此表格显示在表目录中
		\caption{中科院部分推荐期刊及会议}\\
		
		\toprule
		\textbf{序号} & \textbf{简称} \\
		\midrule
		\endfirsthead
		
		% 在这里设计首页以外的表题和表头
		\CPcaption{2}{中科院部分推荐期刊及会议}\\
		\toprule
		\textbf{序号} & \textbf{简称} \\
		\midrule
		\endhead
		
		% 在这里设计首页以外的表尾
		\bottomrule
		\multicolumn{2}{l}{续下页} \\  % 如不希望跨页表尾显示任何内容则注释掉即可
		\endfoot
		
		\bottomrule
		\endlastfoot
		
		1 & JSAC \\
		2 & TMC \\
		3 & TON \\
		1 & TOIT \\
		2 & TOMM \\
		3 & TOSN \\
		4 & CN \\
		5 & TCOM \\
		6 & TWC \\
		2 & CC \\
		3 & TNSM \\
		5 & JNCA \\
		6 & MONET \\
		8 & PPNA \\
		9 & WCMC \\
		11 & IOT \\
		% 1 & SIGCOMM \\
		% 2 & MobiCom \\
		% 3 & INFOCOM \\
		% 4 & NSDI \\
		% 1 & SenSys \\
		% 2 & CoNEXT \\
		% 3 & SECON \\
		% 4 & IPSN \\
		% 5 & MobiSys \\
		% 6 & ICNP \\
		% 7 & MobiHoc \\
		% 8 & NOSSDAV \\
		% 9 & IWQoS \\
		% 10 & IMC \\
		% 1 & JSAC \\
		% 2 & TMC \\
		% 3 & TON \\
		% 1 & TOIT \\
		% 2 & TOMM \\
		% 3 & TOSN \\
		% 4 & CN \\
		% 5 & TCOM \\
		% 6 & TWC \\
		% 2 & CC \\
		% 3 & TNSM \\
		% 5 & JNCA \\
		% 6 & MONET \\
		% 8 & PPNA \\
		% 9 & WCMC \\
		% 11 & IOT \\
		% 1 & SIGCOMM \\
		% 2 & MobiCom \\
		% 3 & INFOCOM \\
		% 4 & NSDI \\
		% 1 & SenSys \\
		% 2 & CoNEXT \\
		% 3 & SECON \\
		% 4 & IPSN \\
		% 5 & MobiSys \\
		% 6 & ICNP \\
		% 7 & MobiHoc \\
		% 8 & NOSSDAV \\
		% 9 & IWQoS \\
		% 10 & IMC \\
	\end{longtable}
	
	\Section{相关工作}

	\begin{equation}
		x^2 + y^2 + z^2
	\end{equation}
	
	\Section{系统模型}

	%% 在“外文资料译文”部分,排版定义、公理、定理、命题、推论、引理、示例、假设需使用与正文一致的环境,如下所示

	\begin{definition}[名称]
		其后,又从全真派逃出,机缘巧合下于古墓遇见小龙女,之后便跟随小龙女练功。他身边有许多红颜知己钟情于他,而他却一心只爱小龙女。
	\end{definition}

	\begin{axiom}[名称]
		其后,又从全真派逃出,机缘巧合下于古墓遇见小龙女,之后便跟随小龙女练功。他身边有许多红颜知己钟情于他,而他却一心只爱小龙女。
	\end{axiom}
	
	\begin{theorem}[名称]
		其后,又从全真派逃出,机缘巧合下于古墓遇见小龙女,之后便跟随小龙女练功。他身边有许多红颜知己钟情于他,而他却一心只爱小龙女。
	\end{theorem}
	
	\begin{proposition}[名称]
		其后,又从全真派逃出,机缘巧合下于古墓遇见小龙女,之后便跟随小龙女练功。他身边有许多红颜知己钟情于他,而他却一心只爱小龙女。
	\end{proposition}
	
	\begin{corollary}[名称]
		其后,又从全真派逃出,机缘巧合下于古墓遇见小龙女,之后便跟随小龙女练功。他身边有许多红颜知己钟情于他,而他却一心只爱小龙女。
	\end{corollary}
	
	\begin{lemma}[名称]
		其后,又从全真派逃出,机缘巧合下于古墓遇见小龙女,之后便跟随小龙女练功。他身边有许多红颜知己钟情于他,而他却一心只爱小龙女。
	\end{lemma}

	\begin{example}[名称]
		其后,又从全真派逃出,机缘巧合下于古墓遇见小龙女,之后便跟随小龙女练功。他身边有许多红颜知己钟情于他,而他却一心只爱小龙女。
	\end{example}

	\begin{assumption}[名称]
		其后,又从全真派逃出,机缘巧合下于古墓遇见小龙女,之后便跟随小龙女练功。他身边有许多红颜知己钟情于他,而他却一心只爱小龙女。
	\end{assumption}
	
	\begin{proof}
		其后,又从全真派逃出,机缘巧合下于古墓遇见小龙女,之后便跟随小龙女练功。他身边有许多红颜知己钟情于他,而他却一心只爱小龙女。
	\end{proof}
	
	\Section{算法设计}

	\begin{algo}[!h]
		\caption{algo环境伪码示例}
		\Input{1) 输入1; 2) 输入2。}
		\Output{输出结果。}

		伪码行1。
		
		\For(\tcc*[f]{循环条件注释1}){循环条件1}{
			伪码行2。
			
			\tcp{注释2}
			伪码行3。
			
			\DoWhile(\tcc*[f]{循环条件注释3}){循环条件2}{
				伪码行4。
			}
			
			\tcc{loop循环}
			\Loop(\tcc*[f]{注释4}){
				循环体1。
			}
			
			\Repeat(\tcc*[f]{循环条件注释5}){循环条件3}{
				循环体2。
			}
			\eIf(\tcc*[f]{条件注释6}){条件语句6}{
				为真,伪码行5。
			}{
				条件为假,伪码行6。\tcp*[f]{repeat循环}
			}
		}
		\textbf{return} 算法结果。
	\end{algo}
	
	\Section{实验}

	itemize示例:

	\begin{itemize}
		\item 杨过少年时期母亲染病而亡,随后他便过着四处流浪的生活。后来遇到郭靖夫妇,便由他们照看。
		\begin{itemize}
			\item 杨过少年时期母亲染病而亡,随后他便过着四处流浪的生活。后来遇到郭靖夫妇,便由他们照看。
			\item 杨过少年时期母亲染病而亡,随后他便过着四处流浪的生活。后来遇到郭靖夫妇,便由他们照看。
		\end{itemize}
		\item 杨过少年时期母亲染病而亡,随后他便过着四处流浪的生活。后来遇到郭靖夫妇,便由他们照看。
	\end{itemize}

	\null

	enumerate示例:

	\begin{enumerate}
		\item 杨过少年时期母亲染病而亡,随后他便过着四处流浪的生活。后来遇到郭靖夫妇,便由他们照看。
		\begin{enumerate}
			\item 杨过少年时期母亲染病而亡,随后他便过着四处流浪的生活。后来遇到郭靖夫妇,便由他们照看。
			\item 杨过少年时期母亲染病而亡,随后他便过着四处流浪的生活。后来遇到郭靖夫妇,便由他们照看。
		\end{enumerate}
		\item 杨过少年时期母亲染病而亡,随后他便过着四处流浪的生活。后来遇到郭靖夫妇,便由他们照看。
	\end{enumerate}
	
	\Section{结论}

	\begin{figure}[!htb]
		\centering
		\includegraphics[width=0.5\linewidth]{洪七公2}
		\captionsetup{list=no}% 阻止此图片显示在图目录中
		\caption{碰都不敢碰啊?小鬼}
	\end{figure}
	
	\begin{figure}[!htb]
		\centering
		\includegraphics[width=0.5\linewidth]{杨过3}
		\captionsetup{list=no}% 阻止此图片显示在图目录中
		\caption{难顶}
	\end{figure}
\end{document}